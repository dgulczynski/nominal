% Opcje klasy 'iithesis' opisane sa w komentarzach w pliku klasy. Za ich pomoca
% ustawia sie przede wszystkim jezyk oraz rodzaj (lic/inz/mgr) pracy.
\documentclass[english, mgr]{iithesis}

\usepackage[utf8]{inputenc}

%%%%% DANE DO STRONY TYTUŁOWEJ
% Niezaleznie od jezyka pracy wybranego w opcjach klasy, tytul i streszczenie
% pracy nalezy podac zarowno w jezyku polskim, jak i angielskim.
% Pamietaj o madrym (zgodnym z logicznym rozbiorem zdania oraz estetyka) recznym
% zlamaniu wierszy w temacie pracy, zwlaszcza tego w jezyku pracy. Uzyj do tego
% polecenia \fmlinebreak.
\englishtitle   {Nominal logic \fmlinebreak for reasoning about terms \fmlinebreak with variable bindings}
\polishtitle    {Logika dziedzinowa do wnioskowania \fmlinebreak o termach z wiązaniem zmiennych}
\polishabstract {Przedstawiamy logikę dziedzinową do wnioskowania o termach z wiązaniem zmiennych. }
\englishabstract{We describe logic for reasoning about terms with variable bindings.}
\author         {Dominik Gulczyński}
\advisor        {dr Piotr Polesiuk}
\date           {\today}                     % Data zlozenia pracy
% Dane do oswiadczenia o autorskim wykonaniu
\transcriptnum  {299391}                     % Numer indeksu
\advisorgen     {dr. Piotra Polesiuka} % Nazwisko promotora w dopelniaczu
%%%%%

%%%%% WLASNE DODATKOWE PAKIETY
\usepackage{amsmath}
\usepackage{amssymb}
\usepackage{stmaryrd} % For \llbracket
\usepackage{macros}
\usepackage{mathpartir}
\usepackage{mathtools}
\usepackage{listings}
\usepackage{semantic}
\usepackage{sourcecodepro}

%%%%% WŁASNE DEFINICJE I POLECENIA
\definecolor{codegreen}{rgb}{0,0.6,0}
\definecolor{codegray}{rgb}{0.5,0.5,0.5}
\definecolor{codepurple}{rgb}{0.58,0,0.82}
\definecolor{backcolour}{rgb}{0.95,0.95,0.92}
\lstdefinestyle{ocamlstyle}{
    commentstyle=\color{codegreen},
    keywordstyle=\color{magenta},
    numberstyle=\tiny\color{codegray},
    stringstyle=\color{codepurple},
    basicstyle=\ttfamily\footnotesize,
    breakatwhitespace=false,
    breaklines=true,
    captionpos=b,
    keepspaces=true,
    numbers=left,
    numbersep=5pt,
    showspaces=false,
    showstringspaces=false,
    showtabs=false,
    tabsize=2
}
\lstset{style=ocamlstyle}
\lstdefinelanguage{OCaml}[]{caml}{
    morekeywords={val, ProofEnv}
}
%%%%%

\begin{document}

%%%%%%%%%%%%%%%%%%%%%%%%%%%%%%%%%%%%%%%%%%%%%%%%%%%%%%%%%%%%%%%%%%%%%%%%%%%%%%%%
\chapter{Introduction}

\section{Problem statement}
\dots

\section{Motivation}
% Przykłady machania rękoma nad wiązaniem zmiennych w innych tekstach jako motywacja do tej pracy
\dots

\section{Related work}
\subsection{Nominal logics \& permutations}

\section{Contributions}
\dots

%%%%%%%%%%%%%%%%%%%%%%%%%%%%%%%%%%%%%%%%%%%%%%%%%%%%%%%%%%%%%%%%%%%%%%%%%%%%%%%%
\chapter{Terms and constraints}
% Okreli odbiorce i do niego pisa

In classical first-order logic, terms are constructed from variables and applications of functional symbols to other terms. This work introduces an extension to terms with expressions closely resembling the syntax of lambda calculus. The aim is to create a flexible framework for reasoning about the lambda calculus and its derivations.

To achieve this goal, we introduce an infinite set of \textit{atoms} (represented by lowercase letters) which correspond to the bound variables in terms, analogous to the variables in lambda calculus. This set is disjoint from the set of variables commonly used in first-order logic, which we will refer to as \textit{variables} (denoted by uppercase letters).

Terms are defined by the following grammar:
\\
\begin{tabular}{rclr}
    $\perm$    & $\deff \permid \mid \permswap{\atomexp}{\atomexp}{\perm}$ \\
    $\atomexp$ & $\deff \perm \apperm \atomv$ \\
    $\term$    & $\deff \atomexp \mid \perm \apperm \termv \mid \tbind{\atomexp} \term \mid \term \tapp \term \mid \symb$
\end{tabular}
\\

It's important to note that terms do not inherently incorporate notions of computation, reduction, or binding. These expressions closely resemble lambda calculus syntax but lack its operational semantics. However, the intuitions associated with these expressions are not baseless. Their practical application is observed in the sublogic of constraints defined on top of terms, used to reason about concepts such as \textit{freshness}, \textit{variable binding}, and \textit{structural} order, as well as their logical model.

Constraints are given by the following grammar:
\\
\begin{tabular}{rclr}
    $\constr$  & $\deff$ & $\atomexp \cfresh \term
                   \bnfor \term \ceq \term
                   \bnfor \term \csheq \term
                   \bnfor \term \cshlt \term$
        & (constraints)
\end{tabular}
\\
with following semantics:
\begin{eqnarray*}
  \atomexp \cfresh \term & \text{---} & \text{
    atom $\atomexp$ is {Fresh} in term $\term$, i.e. does not occur in $\term$
    as a free variable
  } \\
  \term_1 \ceq \term_2 & \text{---} &\text{
    terms $\term_1$ and $\term_2$ are alpha-equivalent
  } \\
  \term_1 \csheq \term_2 & \text{---} &\text{
    terms $\term_1$ and $\term_2$ possess an identical shape,
  } \\
  & & \text{
    i.e. after erasing all atoms, terms $\term_1$ and $\term_2$ would be equal
  } \\
  \term_1 \cshlt \term_2 & \text{---} &\text{
    shape of term $\term_1$ is structurally smaller than the shape of term
    $\term_2$,
  } \\
  & & \text{
    i.e. after erasing all atoms $\term_1$ would be equal to some
    subterm of $\term_2$
  } \\
\end{eqnarray*}

We use metavariable $\cEnv$ to represent finite sets of constraints.


\begin{tabular}{rclr}
$\semterm$ & $\deff$ & $\sematom
               \bnfor n
               \bnfor \stbind \semterm
               \bnfor \semterm \stapp \semterm
               \bnfor \symb$
    & (semantic terms) \\
$\shape$   & $\deff$ & $\shatom
               \bnfor \shbind \shape
               \bnfor \shape \shapp \shape
               \bnfor \symb$
    & (semantic shapes)
\end{tabular}

\begin{eqnarray*}
  \termMdl{\perm \apperm \atomv}{\tmEnv} & = &
    \permMdl{\perm}{\tmEnv}(\tmEnv(\atomv)) \\
  \termMdl{\perm \apperm \termv}{\tmEnv} & = &
    \permMdl{\perm}{\tmEnv}(\tmEnv(\termv)) \\
  \termMdl{\tbind{\atomexp} \term}{\tmEnv} & = &
    \stbind (\termMdl{\term}{\tmEnv} \shiftIdx)
      \subst{\termMdl{\atomexp}{\tmEnv}}{0} \\
  \termMdl{\term_1 \tapp \term_2}{\tmEnv} & = &
    \termMdl{\term_1}{\tmEnv} \stapp \termMdl{\term_2}{\tmEnv} \\
  \termMdl{\symb}{\tmEnv} & = & \symb
\end{eqnarray*}

\begin{eqnarray*}
  \shapeof{\sematom}                     & = & \shatom \\
  \shapeof{n}                            & = & \shatom \\
  \shapeof{\stbind \semterm}             & = & \shbind \shapeof{\semterm} \\1
  \shapeof{\semterm_1 \stapp \semterm_2} & = &
    \shapeof{\semterm_1} \shapp \shapeof{\semterm_2}
\end{eqnarray*}

\begin{eqnarray*}
  \tmEnv \vDash \term_1 \ceq \term_2 & \textrm{iff} &
    \termMdl{\term_1}{\tmEnv} = \termMdl{\term_2}{\tmEnv} \\
  \tmEnv \vDash \atomexp \cfresh \term & \textrm{iff} &
    \termMdl{\atomexp}{\tmEnv} \notin
      \mathsf{FreeAtoms}(\termMdl{\term}{\tmEnv}) \\
  \tmEnv \vDash \term_1 \csheq \term_2 & \textrm{iff} &
    \shapeof{\termMdl{\term_1}{\tmEnv}} = \shapeof{\termMdl{\term_2}{\tmEnv}} \\
  \tmEnv \vDash \term_1 \cshlt \term_2 & \textrm{iff} &
    \shapeof{\termMdl{\term_1}{\tmEnv}} \textrm{ is a strict subshape of }
      \shapeof{\termMdl{\term_2}{\tmEnv}}
\end{eqnarray*}

We write $\tmEnv \vDash \cEnv$ iff for all $\constr \in \cEnv$,
we have $\tmEnv \vDash \constr$.
We write $\cEnv \vDash \constr$ iff
for every $\tmEnv$ such that $\tmEnv \vDash \cEnv$,
we have $\tmEnv \vDash \constr$.

Within this model, we establish the existence of a decidible algorithm for determining whether $C_1, \ldots, C_n \models C_0$, meaning there is a deterministic way to check whether constraints $C_1, \ldots, C_n$ imply $C_0$. This algorithm is presented in the following chapter.


%%%%%%%%%%%%%%%%%%%%%%%%%%%%%%%%%%%%%%%%%%%%%%%%%%%%%%%%%%%%%%%%%%%%%%%%%%%%%%%%
\chapter{Constraint solver}
At the heart of our work lies the Solver, an algorithm designed to resolve constraints.
A high level perspective of the Solver is that it dissects constraints on both sides of the turnstile into irreducible components that are solved easily.

Given a set of assumptions $\constr_1, \dots, \constr_n$, it verifies whether a given goal $\constr_0$ holds.
Technically, the Solver determines whether,
every possible substitution of variables into closed terms in $\constr_0, \constr_1, \dots, \constr_n$,
such that $\constr_1, \dots, \constr_n$ are satisfied, will also satisfy $\constr_0$.

For the sake of convenience and implementation efficiency, the Solver operates
on slightly different constraints compared to those found in formulas and kinds.
The key distinction lies in the use of \textit{shapes} in shape constraints rather than terms.

Solver constraints and shapes are defined by the following grammar: \\
\begin{tabular}{rclr}
  $\sconstr$ & $\deff$ & $\atomexp \cfresh \term
  \bnfor \term \ceq \term
  \bnfor \shape \csheq \shape
  \bnfor \shape \cshlt \shape$
      & (solver constraints) \\
  $\shape$      & $\deff$ & $\shatom
                 \bnfor  \termv
                 \bnfor \shbind \shape
                 \bnfor \shape \tapp \shape
                 \bnfor \symb$
      & (shapes)
\end{tabular}\\
Solver erases atoms from terms in shape constraints,
effectively transforming them from \textit{constraints} to \textit{solver constraints}.

We add another environment $\icEnv$ to distinguish between the potentially-reducible assumptions in $\cEnv$.
For convenience, we will write $\atomv \cneq \atomexp$ instead of $\atomv \cfresh \atomexp$ as it gives a clear intuition of atom freshness implying inequality.
Additionally, when $\atomexp = \pi \atomv$, we will denote $\atomexp \cfresh \term$ to mean $\atomv \cfresh \pi^{-1} \term$.

Irreducible constraints are:
\begin{eqnarray*}
  \atomv_1 \cneq \atomv_2 & \text{---} & \text{
    atoms $\atomv_1$ and $\atomv_2$ are different
  } \\
  \atomv   \cfresh \termv   & \text{---} & \text{
    atom $\atomv$ is {Fresh} in variable $\termv$
  } \\
  \termv_1 \csheq  \termv_2 & \text{---} & \text{
    variables $\termv_1$ and $\termv_2$ posses the same shape
  } \\
  \termv   \csheq  \term    & \text{---} & \text{
    variable $\termv$ has a shape of term $\term$
  } \\
  \term    \cshlt  \termv   & \text{---} & \text{
    term $\term$ strictly subshapes variable $\termv$
  } \\
\end{eqnarray*}

After all the constraints are reduced to such simple constraints
we reduce the goal-constraint and repeat the reduction procedure on
new assumptions and goal. We either arrive at a contradictory environment or
all the assumptions and goal itself are reduced to irreducible constraints, which
is as simple as checking if the goal occurs on the left side of the turnstile:
$$
\inferrule{
  \inferrule{
    \sconstr'' \in \icEnv''
    }{
    \inferrule{
     \cdots
    }{
     \cEnv' ; \icEnv' \vDash \sconstr' \\ \cdots
  }
  }
}{
  \cEnv ; \icEnv \vDash \sconstr
}
$$
Decidability of atom equality plays an important role in the reduce
procedure:
%%%%%%%%%%%%%%%%%%%%%%%%%%%%%%%%%%%%%%%%%%%%%%%%%%%%%%%%%%%%%%%%%%%%%%%%%%%%%%%%
% solver goal eq
%%%%%%%%%%%%%%%%%%%%%%%%%%%%%%%%%%%%%%%%%%%%%%%%%%%%%%%%%%%%%%%%%%%%%%%%%%%%%%%%
$$
\inferrule{
  \cEnv ; \icEnv \vDash \atomv = \perm^{-1} \atomexp
}{
  \cEnv ; \icEnv \vDash \perm \atomv = \atomexp
}
\qquad
\inferrule{
  \atomv \cneq \atomexp_1, \atomv \cneq \atomexp_2, \cEnv ; \icEnv \vDash \atomv     = \atomexp \\
  \atomv \ceq  \atomexp_1, \atomv \cneq \atomexp_2, \cEnv ; \icEnv \vDash \atomexp_2 = \atomexp \\
  \atomv \ceq  \atomexp_2, \cEnv ; \icEnv \vDash \atomexp_1 = \atomexp
}{
  \cEnv ; \icEnv \vDash \atomv = \permswap{\atomexp_1}{\atomexp_2}{}\atomexp
}
$$
$$
\inferrule{
  \cEnv ; \icEnv \vDash \pi \text{ idempotent on } \termv
}{
  \cEnv ; \icEnv \vDash \termv = \pi \termv
}
\qquad
\inferrule{
  \cEnv ; \icEnv \vDash \termv_1 = \pi_1^{-1} \pi_2 \termv_2
}{
  \cEnv ; \icEnv \vDash \pi_1 \termv_1 = \pi_2 \termv_2
}
$$
$$
\inferrule{
  \cEnv ; \icEnv \vDash \atomexp_1 \cfresh \term_2
  \\
  \cEnv ; \icEnv \vDash \term_1 = \permswap{\atomexp_1}{\atomexp_2} \term_2
}{
  \cEnv ; \icEnv \vDash \tbind{\atomexp_1} \term_1 = \tbind{\atomexp_2} \term_2
}
\qquad
\inferrule{
  \cEnv ; \icEnv \vDash \term_1 = \term_2
  \\
  \cEnv ; \icEnv \vDash \term_1' = \term_2'
}{
  \cEnv ; \icEnv \vDash \term_1 \term_1' = \term_2 \term_2'
}
$$
$$\inferrule{
}{
  \cEnv ; \icEnv \vDash \atomv = \atomv
}
\qquad
\inferrule{
}{
  \cEnv ; \icEnv \vDash \termv = \termv
}
\qquad
\inferrule{
}{
  \cEnv ; \icEnv \vDash \symb = \symb
}
$$
%%%%%%%%%%%%%%%%%%%%%%%%%%%%%%%%%%%%%%%%%%%%%%%%%%%%%%%%%%%%%%%%%%%%%%%%%%%%%%%%
% solver permutation idempotent
%%%%%%%%%%%%%%%%%%%%%%%%%%%%%%%%%%%%%%%%%%%%%%%%%%%%%%%%%%%%%%%%%%%%%%%%%%%%%%%%
$$
\inferrule{
  \forall \atomv \in \pi.\;
    \cEnv ; \icEnv \vDash \atomv = \pi \atomv \;\vee\;
    \cEnv ; \icEnv \vDash \atomv \cfresh \termv
  }{
  \cEnv ; \icEnv \vDash \pi \text{ idempotent on } \termv
}
$$
%%%%%%%%%%%%%%%%%%%%%%%%%%%%%%%%%%%%%%%%%%%%%%%%%%%%%%%%%%%%%%%%%%%%%%%%%%%%%%%%
% solver goal {Fresh}
%%%%%%%%%%%%%%%%%%%%%%%%%%%%%%%%%%%%%%%%%%%%%%%%%%%%%%%%%%%%%%%%%%%%%%%%%%%%%%%%
$$\inferrule{
  \atomv_1 \cneq \atomv_2 \in \icEnv
}{
  \cEnv ; \icEnv \vDash \atomv_1 \cfresh \atomv_2
}
\qquad
\inferrule{
  \atomv \cneq \atomexp_1, \atomv \cneq \atomexp_2, \cEnv ; \icEnv \vDash \atomv     \cfresh \atomexp \\
  \atomv \ceq  \atomexp_1, \atomv \cneq \atomexp_2, \cEnv ; \icEnv \vDash \atomexp_1 \cfresh \atomexp \\
                          \atomv \ceq  \atomexp_2 , \cEnv ; \icEnv \vDash \atomexp_2 \cfresh \atomexp
}{
  \cEnv ; \icEnv \vDash \atomv \cfresh \permswap{\atomexp_1}{\atomexp_2}{} \atomexp
}
$$
$$\inferrule{
  \atomv \cfresh \termv \in \icEnv
}{
  \cEnv ; \icEnv \vDash \atomv \cfresh \termv
}
\qquad
\inferrule{
  \atomv \cneq \atomexp_1, \atomv \cneq \atomexp_2, \cEnv ; \icEnv \vDash \atomv     \cfresh \pi \termv \\
  \atomv \ceq  \atomexp_1, \atomv \cneq \atomexp_2, \cEnv ; \icEnv \vDash \atomexp_1 \cfresh \pi \termv \\
                          \atomv \ceq  \atomexp_2 , \cEnv ; \icEnv \vDash \atomexp_2 \cfresh \pi \termv
}{
  \cEnv ; \icEnv \vDash \atomv \cfresh \permswap{\atomexp_1}{\atomexp_2}{\pi} \termv
}
$$
$$\inferrule{
  \atomv \cneq \atomexp, \cEnv ; \icEnv \vDash \atomv \cfresh \term
}{
  \cEnv ; \icEnv \vDash \atomv \cfresh \tbind{\atomexp}{\term}
}
\qquad
\inferrule{
  \cEnv ; \icEnv \vDash \atomv \cfresh \term_1 \\
  \cEnv ; \icEnv \vDash \atomv \cfresh \term_2
}{
  \cEnv ; \icEnv \vDash \atomv \cfresh \term_1 \term_2
}
\qquad
\inferrule{
}{
  \cEnv ; \icEnv \vDash \atomv \cfresh \symb
}
$$
%%%%%%%%%%%%%%%%%%%%%%%%%%%%%%%%%%%%%%%%%%%%%%%%%%%%%%%%%%%%%%%%%%%%%%%%%%%%%%%%
% solver goal shape
%%%%%%%%%%%%%%%%%%%%%%%%%%%%%%%%%%%%%%%%%%%%%%%%%%%%%%%%%%%%%%%%%%%%%%%%%%%%%%%%
$$
\inferrule{
  \termv_1 \csheq \termv_2 \in \icEnv
}{
  \cEnv ; \icEnv \vDash \termv_1 \csheq \termv_2
}
\qquad
\inferrule{
  \termv  \csheq \shape' \in \icEnv \\
  \cEnv ; \icEnv \vDash \shape'  \csheq \shape
}{
  \cEnv ; \icEnv \vDash \termv  \csheq \shape
}
$$
$$\inferrule{
  \cEnv ; \icEnv \vDash \shape_1 \csheq \shape_2
}{
  \cEnv ; \icEnv \vDash \shbind \shape_1 \csheq \shbind \shape_2
}
\qquad
\inferrule{
  \cEnv ; \icEnv \vDash \shape_1 \csheq \shape_2 \\
  \cEnv ; \icEnv \vDash \shape_1' \csheq \shape_2'
}{
  \cEnv ; \icEnv \vDash \shape_1 \shape_1' \csheq \shape_2 \shape_2'
}
\qquad
\inferrule{
}{
  \cEnv ; \icEnv \vDash \symb \csheq \symb
}
$$
%%%%%%%%%%%%%%%%%%%%%%%%%%%%%%%%%%%%%%%%%%%%%%%%%%%%%%%%%%%%%%%%%%%%%%%%%%%%%%%%
% solver goal subshape
%%%%%%%%%%%%%%%%%%%%%%%%%%%%%%%%%%%%%%%%%%%%%%%%%%%%%%%%%%%%%%%%%%%%%%%%%%%%%%%%
$$\inferrule{
  \shape_2 \cshlt \termv \in \icEnv \\
  \cEnv ; \icEnv \vDash \shape_2 \csheq \termv
}{
  \cEnv ; \icEnv \vDash \shape_1 \cshlt \termv
}
\qquad
\inferrule{
  \shape_2 \cshlt \termv \in \icEnv \\
  \cEnv ; \icEnv \vDash \shape_2 \cshlt \termv
}{
  \cEnv ; \icEnv \vDash \shape_1 \cshlt \termv
}
$$
$$\inferrule{
  \cEnv ; \icEnv \vDash \shape_1 \csheq \shape_2
}{
  \cEnv ; \icEnv \vDash \shape_1 \cshlt \shbind \shape_2
}
\qquad
\inferrule{
  \cEnv ; \icEnv \vDash \shape_1 \cshlt \shape_2
}{
  \cEnv ; \icEnv \vDash \shape_1 \cshlt \shbind \shape_2
}
$$
$$\inferrule{
  \cEnv ; \icEnv \vDash \shape_1 \csheq \shape_2
}{
  \cEnv ; \icEnv \vDash \shape_1 \cshlt \shape_2 \shape_2'
}
\qquad
\inferrule{
  \cEnv ; \icEnv \vDash \shape_1 \csheq \shape_2'
}{
  \cEnv ; \icEnv \vDash \shape_1 \cshlt \shape_2 \shape_2'
}
\qquad
\inferrule{
  \cEnv ; \icEnv \vDash \shape_1 \cshlt \shape_2
}{
  \cEnv ; \icEnv \vDash \shape_1 \cshlt \shape_2 \shape_2'
}
\qquad
\inferrule{
  \cEnv ; \icEnv \vDash \shape_1 \cshlt \shape_2'
}{
  \cEnv ; \icEnv \vDash \shape_1 \cshlt \shape_2 \shape_2'
}
$$
%%%%%%%%%%%%%%%%%%%%%%%%%%%%%%%%%%%%%%%%%%%%%%%%%%%%%%%%%%%%%%%%%%%%%%%%%%%%%%%%
% solver assm eq
%%%%%%%%%%%%%%%%%%%%%%%%%%%%%%%%%%%%%%%%%%%%%%%%%%%%%%%%%%%%%%%%%%%%%%%%%%%%%%%%
$$
\inferrule{
  \atomv_1 \cneq \atomv_2 \in \icEnv
}{
  \atomv_1 \ceq \atomv_2, \cEnv ; \icEnv \vDash \sconstr
}
\qquad
\inferrule{
   \cEnv \subst{\atomv_1}{\atomv_2}; \icEnv\subst{\atomv_1}{\atomv_2} \vDash \sconstr\subst{\atomv_1}{\atomv_2}
}{
  \atomv_1 \ceq \atomv_2, \cEnv ; \icEnv \vDash \sconstr
}
$$
$$
\inferrule{
  \atomv \cneq \atomexp_1, \atomv \cneq \atomexp_2, \atomv     \ceq \atomexp, \cEnv ; \icEnv \vDash \sconstr \\
  \atomv \ceq  \atomexp_1, \atomv \cneq \atomexp_2, \atomexp_2 \ceq \atomexp, \cEnv ; \icEnv \vDash \sconstr \\
                           \atomv \ceq  \atomexp_2, \atomexp_1 \ceq \atomexp, \cEnv ; \icEnv \vDash \sconstr
}{
  \atomv \ceq \permswap{\atomexp_1}{\atomexp_1} \atomexp, \cEnv ; \icEnv \vDash \sconstr
}
$$
$$
\inferrule{
  \atomv \ceq \pi^{-1} \atomexp, \cEnv ; \icEnv \vDash \sconstr
}{
  \pi \atomv \ceq \atomexp, \cEnv ; \icEnv \vDash \sconstr
}
\qquad
\inferrule{
}{
   \atomv \ceq \term_1 \term_2, \cEnv ; \icEnv \vDash \sconstr
}
\qquad
\inferrule{
}{
   \atomv \ceq \tbind{\atomexp} \term , \cEnv ; \icEnv \vDash \sconstr
}
\qquad
\inferrule{
}{
   \atomv \ceq \symb , \cEnv ; \icEnv \vDash \sconstr
}
$$
$$
\inferrule{
  \vDash \text{ idempotent on } \termv
}{
   \termv = \pi \termv, \cEnv ; \icEnv \vDash \sconstr
}
\qquad
\inferrule{
  \pi \text{ idempotent on } \termv, \cEnv ; \icEnv \vDash \sconstr
}{
   \termv = \pi \termv, \cEnv ; \icEnv \vDash \sconstr
}
$$
$$
\inferrule{
   \cEnv\subst{\termv}{\term} ; \icEnv\subst{\termv}{\term} \vDash \sconstr\subst{\termv}{\term}
}{
   \termv = \term, \cEnv ; \icEnv \vDash \sconstr
}
\qquad
\inferrule{
   \termv = \pi^{-1} \term, \cEnv ; \icEnv \vDash \sconstr
}{
   \pi \termv = \term, \cEnv ; \icEnv \vDash \sconstr
}
$$
$$
\inferrule{
   \atomexp_1 \cfresh \tbind{\atomexp_2} \term_2,\; \term_1 = \permswap{\atomexp_1}{\atomexp_2}\term_2 , \; \cEnv ; \icEnv \vDash \sconstr
}{
   \tbind{\atomexp_1} \term_1 \ceq \tbind{\atomexp_2} \term_2 , \cEnv ; \icEnv \vDash \sconstr
}
\qquad
\text{Other term constructors trivial}
$$
$$
\inferrule{
  \term_1 \ceq \term_2 , \; \term_1' \ceq \term_2', \;\cEnv ; \icEnv \vDash \sconstr
}{
   \term_1 \term_1' \ceq \term_2 \term_2' , \cEnv ; \icEnv \vDash \sconstr
}
\qquad
\text{Other term constructors trivial}
$$
$$
\inferrule{
  \symb_1 \neq \symb_2
}{
  \symb_1 \ceq \symb_2 , \cEnv ; \icEnv \vDash \sconstr
}
\qquad
\inferrule{
}{
  \symb\ceq \symb , \cEnv ; \icEnv \vDash \sconstr
}
\qquad
\text{Other term constructors trivial}
$$
%%%%%%%%%%%%%%%%%%%%%%%%%%%%%%%%%%%%%%%%%%%%%%%%%%%%%%%%%%%%%%%%%%%%%%%%%%%%%%%%
% solver permutation idempotent
%%%%%%%%%%%%%%%%%%%%%%%%%%%%%%%%%%%%%%%%%%%%%%%%%%%%%%%%%%%%%%%%%%%%%%%%%%%%%%%%
$$
\inferrule{
  (\forall \atomv \in \pi.\;
    \cEnv ; \icEnv \vDash \atomv = \pi \atomv \;\vee\;
    \cEnv ; \icEnv \vDash \atomv \cfresh \termv), \cEnv ; \icEnv \vDash \sconstr
}{
\pi \text{ idempotent on } \termv, \cEnv ; \icEnv \vDash \sconstr
}
$$
%%%%%%%%%%%%%%%%%%%%%%%%%%%%%%%%%%%%%%%%%%%%%%%%%%%%%%%%%%%%%%%%%%%%%%%%%%%%%%%%
% solver assm {Fresh}
%%%%%%%%%%%%%%%%%%%%%%%%%%%%%%%%%%%%%%%%%%%%%%%%%%%%%%%%%%%%%%%%%%%%%%%%%%%%%%%%
$$
\inferrule{
}{
  \atomv \cneq \atomv, \cEnv ; \icEnv \vDash \sconstr
}
\qquad
\inferrule{
  \cEnv ; \{\atomv_1 \cneq \atomv_2\} \cup \icEnv \vDash \sconstr
}{
  \atomv_1 \cneq \atomv_2, \; \cEnv ; \icEnv \vDash \sconstr
}
\qquad
\inferrule{
  \cEnv ; \{\atomv \cfresh \termv\} \cup \icEnv \vDash \sconstr
}{
  \atomv \cfresh \termv, \; \cEnv ; \icEnv \vDash \sconstr
}
$$
$$
\inferrule{
  \atomv \cneq \atomexp_1, \atomv \cneq \atomexp_2, \atomv     \cfresh \atomexp, \cEnv ; \icEnv \vDash \sconstr \\
  \atomv \ceq  \atomexp_1, \atomv \cneq \atomexp_2, \atomexp_2 \cfresh \atomexp, \cEnv ; \icEnv \vDash \sconstr \\
                           \atomv \ceq  \atomexp_2, \atomexp_1 \cfresh \atomexp, \cEnv ; \icEnv \vDash \sconstr
}{
  \atomv \cfresh \permswap{\atomexp_1}{\atomexp_1} \atomexp, \cEnv ; \icEnv \vDash \sconstr
}
$$
$$
\inferrule{
  \atomv \cneq \atomexp_1, \atomv \cneq \atomexp_2, \atomv     \cfresh \termv, \cEnv ; \icEnv \vDash \sconstr \\
  \atomv \ceq  \atomexp_1, \atomv \cneq \atomexp_2, \atomexp_2 \cfresh \termv, \cEnv ; \icEnv \vDash \sconstr \\
                           \atomv \ceq  \atomexp_2, \atomexp_1 \cfresh \termv, \cEnv ; \icEnv \vDash \sconstr
}{
  \atomv \cfresh \permswap{\atomexp_1}{\atomexp_1} \termv, \cEnv ; \icEnv \vDash \sconstr
}
$$
$$
\inferrule{
  \atomv \cfresh \atomexp, \; \cEnv ; \icEnv \vDash \sconstr \\
  \atomv \cfresh \atomexp, \; \atomv \cfresh \term,\;\cEnv ; \icEnv \vDash \sconstr
}{
  \atomv \cfresh \tbind{\atomexp} \term , \cEnv ; \icEnv \vDash \sconstr
}
$$
$$
\inferrule{
  \atomv \cfresh \term_1 , \cEnv ; \icEnv \vDash \sconstr \\
  \atomv \cfresh \term_2 , \cEnv ; \icEnv \vDash \sconstr
}{
  \atomv \cfresh \term_1 \term_2 , \cEnv ; \icEnv \vDash \sconstr
}
$$
$$
\inferrule{
  \cEnv ; \icEnv \vDash \sconstr
}{
  \atomv \cfresh \symb, \cEnv ; \icEnv \vDash \sconstr
}
$$
%%%%%%%%%%%%%%%%%%%%%%%%%%%%%%%%%%%%%%%%%%%%%%%%%%%%%%%%%%%%%%%%%%%%%%%%%%%%%%%%
% solver assm shape eq
%%%%%%%%%%%%%%%%%%%%%%%%%%%%%%%%%%%%%%%%%%%%%%%%%%%%%%%%%%%%%%%%%%%%%%%%%%%%%%%%
$$
\inferrule{
  \cEnv ; \{\termv_1 \csheq \termv_2\} \cup \icEnv \vDash \sconstr
}{
  \termv_1 \csheq \termv_2, \cEnv ; \icEnv \vDash \sconstr
}
\qquad
\inferrule{
  \cEnv ; \{\termv \csheq \shape\} \cup \icEnv \vDash \sconstr
}{
  \termv \csheq \shape,\; \cEnv ; \icEnv \vDash \sconstr
}
$$
$$
\inferrule{
  \cEnv ; \icEnv \vDash \sconstr
}{
  \atomv_1 \csheq \atomv_2, \cEnv ; \icEnv \vDash \sconstr
}
\qquad
\text{Other term constructors trivial}
$$
$$
\inferrule{
  \term_1 \csheq \term_2, \cEnv ; \icEnv \vDash \sconstr
}{
  \shbind\term_1 \csheq \shbind\term_2, \cEnv ; \icEnv \vDash \sconstr
}
\qquad
\text{Other term constructors trivial}
$$
$$
\inferrule{
  \term_1  \csheq \term_2 , \cEnv ; \icEnv \vDash \sconstr \\
  \term_1' \csheq \term_2', \cEnv ; \icEnv \vDash \sconstr
}{
  \term_1 \term_1' \csheq \term_2\term_2', \cEnv ; \icEnv \vDash \sconstr
}
\qquad
\text{Other term constructors trivial}
$$
$$
\inferrule{
  \symb_1 \neq \symb_2
}{
  \symb_1 \csheq \symb_2 , \cEnv ; \icEnv \vDash \sconstr
}
\qquad
\inferrule{
}{
  \symb \csheq \symb , \cEnv ; \icEnv \vDash \sconstr
}
\qquad
\text{Other term constructors trivial}
$$
%%%%%%%%%%%%%%%%%%%%%%%%%%%%%%%%%%%%%%%%%%%%%%%%%%%%%%%%%%%%%%%%%%%%%%%%%%%%%%%%
% solver assm shape eq
%%%%%%%%%%%%%%%%%%%%%%%%%%%%%%%%%%%%%%%%%%%%%%%%%%%%%%%%%%%%%%%%%%%%%%%%%%%%%%%%
$$
\inferrule{
  \cEnv ; \{\term \cshlt \termv\} \cup \icEnv \vDash \sconstr
}{
  \term \cshlt \termv, \cEnv ; \icEnv \vDash \sconstr
}
$$
$$
\inferrule{
  \term_1 \csheq \term_2, \cEnv ; \icEnv \vDash \sconstr \\
  \term_1 \cshlt \term_2, \cEnv ; \icEnv \vDash \sconstr
}{
  \term_1 \cshlt \shbind \term_2, \cEnv ; \icEnv \vDash \sconstr
}
$$
$$
\inferrule{
  \term_1 \csheq \term_2, \cEnv ; \icEnv \vDash \sconstr \\
  \term_1 \csheq \term_2', \cEnv ; \icEnv \vDash \sconstr \\
  \term_1 \cshlt \term_2, \cEnv ; \icEnv \vDash \sconstr \\
  \term_1 \cshlt \term_2', \cEnv ; \icEnv \vDash \sconstr
}{
  \term_1 \cshlt \term_2 \term_2', \cEnv ; \icEnv \vDash \sconstr
}
$$
$$
\inferrule{
}{
  \term \cshlt \atomexp, \cEnv ; \icEnv \vDash \sconstr
}
\qquad
\inferrule{
}{
  \term \cshlt \symb, \cEnv ; \icEnv \vDash \sconstr
}
$$

TODO: explain what is $\{\sconstr\} \cup \icEnv$
Additional rule for ariving in contradictory $\icEnv$:
$$
\inferrule{
}{
  \cEnv ; \lightning \vDash \sconstr
}
$$


Define state of the solver by triple $(\cEnv, \icEnv, \sconstr_0)$ and such
ordering of the states:
\begin{enumerate}
  \item Number of distinct variables in $\cEnv$, $\icEnv$,$ \sconstr_0$.
  \item Depth of $\sconstr_0$.
  \item Number of assumptions of given depth in $\cEnv$ and $\icEnv$.
  \item Number of assumptions of given depth in $\cEnv$.
\end{enumerate}

Then by analysing each rule we can see the reductions always arrive in a smaller
state.

\section{Implementation}
Environment $\icEnv$ is a quintuple
$({NeqAtoms}_{\icEnv}
, {Fresh}_{\icEnv}
, {VarShape}_{\icEnv}
, {Shape}_{\icEnv}
, {Subshape}_{\icEnv}
)$ where: \\
$NeqAtoms$ is a set of pairs of atoms that we know are different, \\
$Fresh$ is a mapping from atoms to variables that we know the atom is {Fresh} in,\\
$VarShape$ is a mapping from variables to shape-representative variables (i.e. all variables that are mapped in $VarShape$ to the same variable are of the same shape),\\
$Shape$ is a mapping from shape-representative variables to the shape that we know this variable must have,\\
$SubShape$ is a mapping from shape-representative variables to sets of shapes that we know this variable must supershape.
\\
\\
\newcommand{\shrep}[2][\icEnv]{\ensuremath{ #2_{#1}}}
\newcommand{\shenv}[2][\icEnv]{\ensuremath{ |#2|_{#1}}}
We can now define a way to compute the shape-representative variable:
$$
\shrep{\termv} :=
     \begin{cases}
      \termv          &\text{if } {VarShape}_{\icEnv}(\termv) = \emptyset \\
       \shrep{\termv'} &\text{if } {VarShape}_{\icEnv}(\termv) = \termv'
     \end{cases}
$$
And shape-reconstruction:
\begin{eqnarray*}
  \shenv{\termv}                 &:=&
  \begin{cases}
    \shenv{\termv'} &\text{if } {VarShape}_{\icEnv}(\termv) = \termv' \\
    \shape          &\text{if } {Shape}_{\icEnv}(\termv) = \shape \\
    \termv          &\text{otherwise}
  \end{cases} \\
  \shenv{\shatom}           &:=& \shatom \\
  \shenv{\shbind \shape}    &:=& \shbind \shenv{\shape} \\
  \shenv{\shape_1 \shape_2} &:=& \shenv{\shape_1} \shenv{\shape_2}\\
  \shenv{\symb}             &:=& \symb \\
  \shenv{\term}             &:=& \shenv{|\term|}
\end{eqnarray*}
\newcommand{\occurs}[2]{\ensuremath{ {#1}\text{ occurs in }{#2}}}
\newcommand{\stxoccurs}[2]{\ensuremath{ {#1}\text{ occurs syntactically in }{#2}}}
\newcommand{\pluseq}{\mathrel{+}=}
\newcommand{\minuseq}{\mathrel{-}=}
Now we can easily check for irreducible constraints in $\icEnv$:

\begin{eqnarray*}
  (\atomv_1 \cneq \atomv_2) \in \icEnv &:=& (\atomv_1 \cneq \atomv_2) \in {NeqAtoms}_{\icEnv} \\
  (\atomv \cfresh \termv) \in \icEnv &:=& \termv \in {Fresh}_{\icEnv}(\atomv) \\
  (\termv_1 \csheq \termv_2) \in \icEnv &:=& \shenv{\termv_1} \ceq \shenv{\termv_2} \\
  (\termv \csheq \shape) \in \icEnv &:=& \shape = {Shape}_{\icEnv}(\shrep{\termv})\\
  (\shape \cshlt \termv) \in \icEnv &:=& \shape \in {SubShape}_{\icEnv}(\shrep{\termv})
\end{eqnarray*}

Now we can define rules for the special occurs check:
$$
\inferrule{
  \stxoccurs{\shrep{\termv}}{\shenv{\shape}}
}{
  \icEnv \vDash \occurs{\termv}{\shape}
}
$$
$$
\inferrule{
  \stxoccurs{\shrep{\termv'}}{\shenv{\shape}} \\
  (\shape' \cshlt \termv') \in \icEnv  \\
  \icEnv \vDash \occurs{\termv}{\shape'}
}{
  \icEnv \vDash \occurs{\termv}{\shape}
}
$$
And finally, the rules for $\sconstr \cup \icEnv$.
Note that we are using the meta-field of $Assumptions$ to indicate that some of the
assumptions in $\icEnv$ are no longer "simple" and escape from $\icEnv$ back to
$\cEnv$ to be broken up by the \textit{Solver}.

$$
 \{\atomv \cfresh \termv\} \cup \icEnv := \icEnv[{Fresh}(\atomv) \pluseq \termv]
$$
$$
\{\atomv \cneq \atomv'\} \cup \icEnv :=
  \begin{cases}
    \lightning &\text{if } \atomv \ceq \atomv'  \\
    \icEnv[{NeqAtoms} \pluseq (\atomv \cneq \atomv')] &\text{otherwise.}
  \end{cases}
$$

\begin{eqnarray*}
\{\termv \csheq \shape\} \cup \icEnv & := &
  \begin{cases}
    \lightning     &\text{if } \icEnv \vDash \occurs{\termv}{\shape} \\
    \icEnv' &\text{otherwise.}\\
  \end{cases} \\
  \text{where } \icEnv' & = & \icEnv .{Symbols} \{\shrep{\termv} \leadsto \shenv{\shape} \}\\
                        &   & \quad .{Subshapes}\{\shrep{\termv} \leadsto \shenv{\shape} \} \\
                        &   & \quad .{Shape}    \{\shrep{\termv} \leadsto \shenv{\shape} \}
\end{eqnarray*}

\begin{eqnarray*}
\{\termv \csheq \termv'\} \cup \icEnv & := &
  \begin{cases}
    \icEnv     &\text{if } \shrep{\termv} \ceq \shrep{\termv'} \\
    \icEnv &\text{if } \shenv{\termv} \ceq \shenv{\termv'} \\
    \lightning     &\text{if } \occurs{\shrep{\termv}}{\shenv{\termv'}} \\
    \lightning     &\text{if } \occurs{\shrep{\termv'}}{\shenv{\termv}} \\
    \icEnv' &\text{otherwise.}\\
  \end{cases} \\
  \text{where } \icEnv' & = & \icEnv .{Symbols}     \{\shrep{\termv} \leadsto \shrep{\termv'}\}\\
                        &   & \quad .{Subshapes}    \{\shrep{\termv} \leadsto \shrep{\termv'}\} \\
                        &   & \quad .{TransferShape}\{\shrep{\termv} \leadsto \shrep{\termv'}\} \\
                        &   & \quad [\: {Shape}    \minuseq (\shrep\termv) \\
                        &   & \quad ,   {SubShape} \minuseq (\shrep\termv) \\
                        &   & \quad ,   {VarShape} \pluseq  (\shrep\termv \mapsto \shrep\termv') \\
                        &   & \quad ]
\end{eqnarray*}
$$
\icEnv.{Symbols}\{\termv \leadsto \shape\} :=
  \begin{cases}
    \icEnv[{Symbols} \minuseq \termv, {Assumptions} \pluseq \text{symbol } \shape]    &\text{if } \shrep{\termv} \in \icEnv.{Symbols} \\
    \icEnv &\text{otherwise.}\\
  \end{cases} \\
$$
$$
\icEnv.{Shape}\{\termv \leadsto \shape\} :=
  \begin{cases}
    \icEnv[{Assumptions} \pluseq (\shape \csheq \shape')]    &\text{if } {Shape}_{\icEnv}(\shrep\termv) = \shape' \\
    \icEnv[{Shapes}      \pluseq (\termv \mapsto \shape)] &\text{otherwise.}\\
  \end{cases} \\
$$
$$
\icEnv.{SubShapes}\{\termv \leadsto \shape\} :=
  \icEnv[{Assumptions} \pluseq {Subshapes}_{\icEnv}(\termv) \cshlt \shape]
$$
$$
\icEnv.{TransferShape}\{\termv \leadsto \termv'\} :=
  \begin{cases}
    \icEnv.{Shape}\{termv' \leadsto  \shape'\}    &\text{if } {Shape}_{\icEnv}(\shrep\termv) = \shape \\
    \icEnv &\text{otherwise.}\\
  \end{cases} \\
$$
$$
\icEnv\{\termv \mapsto \term\} := \{\termv \csheq \shenv{\term} \} \cup \icEnv.{Fresh}\{\termv \mapsto \term\}
$$
$$
\icEnv.{Fresh}\{\termv \mapsto \term\} :=
 \icEnv[{Fresh}.map(\text{fun } (\atomv \cfresh \mathbb{\termv}) \mapsto \atomv \cfresh (\mathbb{\termv} \setminus \{ \termv\})] \;
 \cup \bigcup_{\substack{(\atomv \cfresh \mathbb{\termv}) \in {Fresh}_{\icEnv} \\ \termv \in \mathbb{\termv}}}
    \{ \atomv \cfresh \term \}
$$
$$
\icEnv\{\atomv \mapsto \atomv'\} := \icEnv.{Fresh}\{\atomv \mapsto \atomv'\}.{NeqAtoms}\{\atomv \mapsto \atomv'\}]
$$
$$
\icEnv.{Fresh}\{\atomv \mapsto \atomv'\} := \icEnv[{Fresh} \minuseq \atomv][{Fresh} \pluseq \{\atomv' \cfresh \icEnv.{Fresh}(\atomv)\}]
$$
$$
\icEnv.{NeqAtoms}\{\atomv \mapsto \atomv'\} :=
\icEnv[{NeqAtoms} = \emptyset] \;
 \cup \bigcup_{ (\atomv_1 \cneq \atomv_2) \in {NeqAtoms}_{\icEnv}  }
    \{ \atomv_1\{\atomv \mapsto \atomv'\} \cneq \atomv_2\{\atomv \mapsto \atomv'\}\}
$$

%%%%%%%%%%%%%%%%%%%%%%%%%%%%%%%%%%%%%%%%%%%%%%%%%%%%%%%%%%%%%%%%%%%%%%%%%%%%%%%%
\chapter{Higher Order Logic}

On top of the sublogic of constraints, we build a higher-order logic.
Due to the involvement of atoms, terms, binders, and constraints,
we introduce kinds to ensure that the formulas we deal with \textit{make sense}.

%%%%%%%%%%%%%%%%%%%%%%%%%%%%%%%%%%%%%%%%%%%%%%%%%%%%%%%%%%%%%%%%%%%%%%%%%%%%%%%
\section{Kinds}

\begin{tabular}{rclr}
$\kind$ & $::=$ & $\kProp
            \bnfor \kind \karrow \kind
            \bnfor \kForallAtom{\atomv} \kind
            \bnfor \kForallTerm{\termv} \kind
            \bnfor \kGuard{\constr} \kind$
    & (kinds)
\end{tabular}

\begin{tabular}{rclr}
$\formphi \ofkind$ & $\kProp$ & ---  $\formphi$ is a propositional formula. \\
$\formphi \ofkind$ & $\kind_1 \karrow \kind_2$ & ---  $\formphi$ is function that takes a formula of kind $\kind_1$, \\
  & &  and produces a formula of kind $\kind_2$. \\
$\formphi \ofkind$ & $\kForallAtom{\atomv} \kind$ & ---  $\formphi$ is function that takes an an atom expression, \\
  & &  binds it to $\atomv$ and produces a formula of kind $\kind$.\\
$\formphi \ofkind$ & $\kForallTerm{\termv} \kind$ & ---  $\formphi$ is function that takes a term, \\
  & &  binds it to $\termv$ and produces a formula of kind $\kind$.\\
$\formphi \ofkind$ & $\kGuard{\constr} \kind$ & ---  $\formphi$ is a formula of kind $\kind$ as long as $\constr$ is satisfied.
\end{tabular}
\\ \\
Notice that as constraints occur in kinds, we cannot simply give functions
from atoms some kind ${Atom}\karrow\kind$, but we must know \textit{which} atom
is bound there, to substitute for it in $\kind$ the same way we substitute
that atom for an atom expression in the function body when applying it to the formula.
The \textit{guarded kind} $\kGuard{\constr} \kind$ is most importantly used in
kinding of the fixpoint formulas, which we will explain in later sections.

\section{Subkinding}
Kinding relation is relaxed through the \textit{subkinding},
a relation that is naturally reflexive and transitive:
$$
\inference{
}{
  \cEnv \vdash \kind \subkind \kind
}
\qquad
\inference{
  \cEnv \vdash \kind_1 \subkind \kind_2 &
  \cEnv \vdash \kind_2 \subkind \kind_3
}{
  \cEnv \vdash \kind_1 \subkind \kind_3
}
$$
Universally quantified kinds only subkind if they are quantified over the same name:
$$
\inference{
  \cEnv \vdash \kind_1 \subkind \kind_2
}{
  \cEnv \vdash \kForallAtom{\atomv} \kind_1 \subkind \kForallAtom{\atomv} \kind_2
}
\qquad
\inference{
  \cEnv \vdash \kind_1 \subkind \kind_2
}{
  \cEnv \vdash \kForallTerm{\termv} \kind_1 \subkind \kForallTerm{\termv} \kind_2
}
$$
Function kind is contravariant to the subkinding relation on the left argument:
$$
\inference{
  \cEnv \vdash \kind_1' \subkind \kind_1 &
  \cEnv \vdash \kind_2 \subkind \kind_2'
}{
  \cEnv \vdash \kind_1 \karrow \kind_2 \subkind \kind_1' \karrow \kind_2'
}
$$
Constraints that are solved through $\vDash$ relation can be dropped:
$$
\inference{
  \cEnv \vDash \constr
}{
  \cEnv \vdash \kGuard{\constr}\kind \subkind \kind
}
$$
And constraints can be moved to the enviroment from the right-hand side:
$$
\inference{
  \cEnv, \constr \vdash \kind_1 \subkind \kind_2
}{
  \cEnv \vdash \kind_1 \subkind \kGuard{\constr}\kind_2
}
$$
Note that there is no structural subkinding rule for guarded kinds like
$$
\inference{\cEnv \vdash \kind_1 \subkind \kind_2}{
  \cEnv \vdash \kGuard{\constr} \kind_1 \subkind \kGuard{\constr} \kind_2
}[\ensuremath{\times}]
$$
Such a rule can be derived from both subkinding rules for guarded kind,
transitivity, and weakening.

%%%%%%%%%%%%%%%%%%%%%%%%%%%%%%%%%%%%%%%%%%%%%%%%%%%%%%%%%%%%%%%%%%%%%%%%%%%%%%%
\section{Formulas}
Formulas include standard connectives (of kind $\kProp$):

\begin{tabular}{rrlr}
$\formphi$ & $::=$ & $\bot
               \bnfor \top
               \bnfor \formphi \vee \formphi
               \bnfor \formphi \wedge \formphi
               \bnfor \formphi \fImp \formphi
               \bnfor \ldots $ & (formulas)
\end{tabular}
\\ \\
Quantification over atoms and terms (on formulas of kind $\kProp$):

\begin{tabular}{rrlr}
$\formphi$ & $::=$ & $\ldots
               \bnfor \fForallAtom{\atomv} \formphi
               \bnfor \fForallTerm{\termv} \formphi
               \bnfor \fExistsAtom{\atomv} \formphi
               \bnfor \fExistsTerm{\termv} \formphi
               \bnfor \ldots$
    & (formulas)
\end{tabular}
\\ \\
Constraints, guards, and propositional variables:

\begin{tabular}{rrlr}
$\formphi$ & $::=$ & $\ldots
               \bnfor \fConstr{\constr}
               \bnfor \fCAnd{\constr} \formphi
               \bnfor \fCImp{\constr} \formphi
               \bnfor \propv
               \bnfor \ldots$  (formulas)
\end{tabular}
$$
\inference{
}{
  \cEnv; \kEnv \vdash \fConstr{\constr} \ofkind \kProp
}
\qquad
\inference{
  \cEnv,\constr; \kEnv \vdash \formphi \ofkind \kProp
}{
  \cEnv; \kEnv \vdash \fCAnd{\constr} \formphi \ofkind \kProp
}
\qquad
\inference{
  \cEnv,\constr; \kEnv \vdash \formphi \ofkind \kProp
}{
  \cEnv; \kEnv \vdash \fCImp{\constr} \formphi \ofkind \kProp
}
\qquad
\inference{
  (\propv \ofkind \kind) \in \kEnv
}{
  \cEnv; \kEnv \vdash  \propv \ofkind \kind
}
$$
Propositional variables, functions and applications:

\begin{tabular}{rrlr}
$\formphi$ & $::=$ & $\ldots
               \bnfor \fLamAtom{\atomv} \formphi
               \bnfor \fLamTerm{\termv} \formphi
               \bnfor \fLamForm{\propv}{\kind} \formphi
               \bnfor \formphi \fAppAtom{\atomexp}
               \bnfor \formphi \fAppTerm{\term}
               \bnfor \formphi \fApp \formphi
               \bnfor \ldots$
    & (formulas)
\end{tabular}
$$
\inference{
  \cEnv; \kEnv \vdash \formphi \ofkind \kind
}{
  \cEnv; \kEnv \vdash \fLamAtom{\atomv} \formphi \ofkind \kForallAtom{\atomv}\kind
}
\qquad
\inference{
  \cEnv; \kEnv \vdash \formphi \ofkind \kind
}{
  \cEnv; \kEnv \vdash \fLamTerm{\termv} \formphi \ofkind \kForallTerm{\termv}\kind
}
\qquad
\inference{
  \cEnv; \kEnv, \propv \ofkind \kind_1 \vdash \formphi \ofkind \kind_2
}{
  \cEnv; \kEnv \vdash \fLamForm{\propv}{\kind_1} \formphi \ofkind \kind_1 \karrow \kind_2
}
$$
$$
\inference{
  \cEnv;\kEnv\vdash \formphi \ofkind \kForallAtom{\atomv}\kind
}{
  \cEnv;\kEnv\vdash \formphi \fAppAtom{\atomexp} \ofkind \kind \subst{\atomv}{\atomexp}
}
\qquad
\inference{
  \cEnv;\kEnv\vdash \formphi \ofkind \kForallTerm{\termv}\kind
}{
  \cEnv;\kEnv\vdash \formphi \fAppTerm{\term} \ofkind \kind\subst{\termv}{\term}
}
\qquad
\inference{
  \cEnv; \kEnv \vdash \formphi_1 \ofkind \kind' \karrow \kind &
  \cEnv; \kEnv \vdash \formphi_2 \ofkind \kind'
}{
  \cEnv;\kEnv\vdash \formphi_1 \fApp \formphi_2 \ofkind \kind
}
$$
%%%%%%%%%%%%%%%%%%%%%%%%%%%%%%%%%%%%%%%%%%%%%%%%%%%%%%%%%%%%%%%%%%%%%%%%%%%%%%%
\section{Fixpoint}
\newcommand{\fix}[3]{\ensuremath{\text{fix }#1(#2)\ofkind#3=}}
And finish the definition of formulas with \textit{fixpoint} function:

\begin{tabular}{rrlr}
$\formphi$ & $::=$ & $\ldots
               \bnfor \fix{\propv}{\termv}{\kind}{\formphi} $
    & (formulas)
\end{tabular}
$$
\inference{
  \cEnv;\kEnv, (\propv \ofkind \kForallTerm{Y} \kGuard{Y \cshlt \termv}{\kind\subst{\termv}{Y}})\vdash \formphi \ofkind \kind
}{
  \cEnv;\kEnv\vdash (\fix{\propv}{\termv}{\kind}{\formphi}) \ofkind \kForallTerm{\termv}{\kind}
}
$$
The fixpoint constructor allows us to express \textit{recursive} predicates over terms,
but only such that the recursive applications are on structurally smaller terms,
which we express in the kinding rule through the kinding $(\propv \ofkind \kForallTerm{Y} \kGuard{Y \cshlt \termv}{\kind\subst{\termv}{Y}})$.
To evaluate a fixpoint function applied to a term, simply substitute the bound
variable with the given term and replace recursive calls inside the fixpoint's body with the fixpoint itself.

$$
(\fix{\propv}{\termv}{\kind}{\formphi})\fApp\term
\equiv
\formphi\subst{\termv}{\term}\subst{\propv}{(\fix{\propv}{\termv}{\kind}{\formphi})}
$$
Because the applied term is finite
and we always recurse on structurally smaller terms,
the final formula after all substitutions must also be finite
--— thanks to the semantics of constraints and kinds.

To familiarize the reader with the fixpoint formulas,
we present how Peano arithmetic can be modeled in our logic.
Given symbols $0$ and $S$ for natural number construction,
one can write a predicate that a term models some natural number:
$$
\fix{Nat}{N}{\kProp}{(N \ceq 0) \vee (\fExistsTerm{M} \fCAnd{N \ceq {S \tapp M}} ({Nat}\fAppTerm{M}))}
$$
Notice how the constraint $(N \ceq {S \tapp M})$ guards the recursive call to $Nat$,
ensuring that constraint $(M \cshlt N)$ will be satisfied during kind checking of
$({{Nat}\fAppTerm{M}})$ in the kind derivation of the whole formula
$({Nat} \ofkind \kForallTerm{N}{\kProp})$.

Similarly, we can define addition: \\ \\
$\fix{PlusEq}{N}{\kForallTerm{M}\kForallTerm{K}\kProp}{\fLamTerm{M}\fLamTerm{K}}$ \\
$\text{ }\qquad
  (\fCAnd{N \ceq 0}{(M \ceq K)}) \vee
   (\fExistsTerm{N', K'}\fCAnd{N \ceq {S \tapp N'}}\fCAnd{K \ceq {S \tapp K'}}{({PlusEq}\fApp{N'}\fApp{M}\fApp{K'})}
   )
$
\\
TODO: Write how $N$ is treated differently from $M$ and $K$?
\\
See more interesting examples of fixpoints usage in the chapter on STLC.

%%%%%%%%%%%%%%%%%%%%%%%%%%%%%%%%%%%%%%%%%%%%%%%%%%%%%%%%%%%%%%%%%%%%%%%%%%%%%%%
\section{Proof theory}

\newcommand{\rel}[2][\Gamma;C]{\ensuremath{#1\vdash#2}}
\newcommand{\types}[3][\Gamma]{\rel[#1]{#2 : #3}}
\newcommand{\interp}[2][\tmEnv]{\left\llbracket {#2} \right\rrbracket_{#1}}
\newcommand{\arr}{\rightarrow }
% \newcommand{\inference}[2]{\inferrule{ #1}{#2}}
\newcommand{\all}[1][x]{\ensuremath{\forall #1.\:}}
\newcommand{\exi}[1][x]{\ensuremath{\exists #1.\:}}
\newcommand{\karr}{\Rightarrow }
\newcommand{\lam}[1][x]{\lambda{#1}.\;}
\newcommand{\jgmnt[2]}[\cEnv;\Theta]{\ensuremath{#1 \vdash #2}}
\newcommand{\cjgmnt[2]}[\cEnv]{\ensuremath{#1 \vDash #2}}
\newcommand{\fv[1]}[\cEnv;\Theta]{\ensuremath{\operatorname{FV}(#1)}}

Finally, we can define proof-theoretic rules.
Starting with inference rules for assumption,
we can already define its constraint-sublogic analogues that employ the solver.
And while the $\vdash$ relation we define is purely syntactic,
we can still use semantic $\vDash$ because of its decidability.
$$
  \inference{
    \formphi \in \Theta
  }{
    \jgmnt[\cEnv;\Theta]{\formphi}
  }[(\ensuremath{Assumption})]
  \qquad
  \inference{
    \cjgmnt[\cEnv]{\constr}
  }{
    \jgmnt[\cEnv;\Theta]{\constr}
  }[(\ensuremath{constr^i})]
$$
Again, for \textit{ex falso}, we define an analogous proof constructor for dealing with a contradictory
constraint environment.
Note that there are many constraints that can be used as $\bot_\constr$, i.e.
constraints that are always false, and the solver will only \textit{prove} them
if we supply it with contradictory assumptions.
$$
  \inference{
    \jgmnt[\cEnv;\Theta]{\bot}
  }{
    \jgmnt[\cEnv;\Theta]{\formphi}
  }[(\ensuremath{\bot^e})]
  \qquad
  \inference{
    \cjgmnt[\cEnv]{\bot_\constr}
    }{
    \jgmnt[\cEnv;\Theta]{\formphi}
  }[(\ensuremath{constr^e})]
$$
Inference rules for implication are standard, and the reason we present them here
is not to bore the reader, but to point out the similarities to their constraint analogues.
$$
  \inference{
    \jgmnt[\cEnv;\Theta,\formphi_1]{\formphi_2}
  }{
    \jgmnt[\cEnv;\Theta]{\formphi_1 \fImp \formphi_2}
  }[(\ensuremath{\fImp^i})]
  \qquad
  \inference{
    \jgmnt[\cEnv_1;\Theta_1]{\formphi_1} &
    \jgmnt[\cEnv_2;\Theta_2]{\formphi_1 \fImp \formphi_2}
    }{
    \jgmnt[\cEnv_1 \cup \cEnv_2;\Theta_2 \cup \Theta_2]{\formphi_2}
  }[(\ensuremath{\fImp^e})]
$$
$$
  \inference{
    \jgmnt[\cEnv, \constr;\Theta]{\formphi}
  }{
    \jgmnt[\cEnv;\Theta]{\fCImp{\constr}\formphi}
  }[(\ensuremath{\fCImp{\cdot}^i})]
  \qquad
  \inference{
    \jgmnt[\cEnv_1;\Theta_1]{\constr} &
    \jgmnt[\cEnv_2;\Theta_2]{\fCImp{\constr}\formphi}
    }{
    \jgmnt[\cEnv_1 \cup \cEnv_2;\Theta_2 \cup \Theta_2]{\formphi}
  }[(\ensuremath{\fCImp{\cdot}^e})]
$$
Notice that in the case of constraint-and-guard, the rule for elimination is restricted
to only formulas of kind $\kProp$.
This is due to the nature of the guard --- if we want to eliminate it,
we can only do so with formulas that \textit{make sense} on their own,
without that $\constr$ guard.
$$
  \inference{
    \jgmnt[\cEnv_1;\Theta_1]{\formphi_1} &
    \jgmnt[\cEnv_2;\Theta_2]{\formphi_2}
  }{
    \jgmnt[\cEnv_1 \cup \cEnv_2;\Theta_2 \cup \Theta_2]{\formphi_1 \wedge \formphi_2}
  }[(\ensuremath{\wedge^i})]
  \qquad
  \inference{
    \jgmnt[\cEnv;\Theta]{\formphi_1 \wedge \formphi_2}
    }{
    \jgmnt[\cEnv;\Theta]{\formphi_1}
  }[(\ensuremath{\wedge^e_1})]\qquad
  \inference{
    \jgmnt[\cEnv;\Theta]{\formphi_1 \wedge \formphi_2}
    }{
    \jgmnt[\cEnv;\Theta]{\formphi_2}
  }[(\ensuremath{\wedge^e_2})]
$$
$$
  \inference{
    \cjgmnt[\cEnv]{\constr} &
    \jgmnt[\cEnv, \constr;\Theta]{\formphi}
  }{
    \jgmnt[\cEnv;\Theta]{\fCAnd{\constr}\formphi}
  }[(\ensuremath{\fCAnd{\cdot}^i})]
  \qquad
  \inference{
    \jgmnt[\cEnv;\Theta]{\fCAnd{\constr}\formphi}
    }{
    \jgmnt[\cEnv;\Theta]{\constr}
  }[(\ensuremath{\fCAnd{\cdot}^e_1})]\qquad
  \inference{
    \jgmnt[\cEnv]{\fCAnd{\constr}\formphi} &
    \jgmnt[\cEnv;\Theta]{\formphi: \kProp}
    }{
    \jgmnt[\cEnv;\Theta]{\formphi}
  }[(\ensuremath{\fCAnd{\cdot}^e_2})]
$$
Inference rules for disjunction and quantifiers are rather straightforward.
As one would expect, we restrict the generalized name to be \textit{fresh} in the environment (it may not occur in any of the assumptions),
and the names given to witnesses of existential quantification must also be \textit{fresh}.
Rules for quantifiers always come in pairs --- one for the atoms and one for the variables.
$$
  \inference{
    \jgmnt[\cEnv;\Theta]{\formphi_1}
    }{
    \jgmnt[\cEnv;\Theta]{\formphi_1 \vee \formphi_2}
  }[(\ensuremath{\vee^i_1})]
  \qquad
  \inference{
    \jgmnt[\cEnv;\Theta]{\formphi_2}
  }{
    \jgmnt[\cEnv;\Theta]{\formphi_1 \vee \formphi_2}
  }[(\ensuremath{\vee^i_2})]
  \qquad
  \inference{
    \jgmnt[\cEnv;\Theta]{\formphi_1 \vee \formphi_2} \\
    \jgmnt[\cEnv;\Theta,\formphi_1]{\psi} &
    \jgmnt[\cEnv;\Theta,\formphi_2]{\psi}
  }{
    \jgmnt[\cEnv;\Theta]{\psi}
  }[(\ensuremath{\vee^e})]
$$
$$
  \inference{
    \atomv \notin \fv[\cEnv;\Theta] &
    \jgmnt[\cEnv;\Theta]{\formphi}
  }{
    \jgmnt[\cEnv;\Theta]{\fForallAtom{\atomv}\formphi}
  }[(\ensuremath{{\fForallAtom{}}^i})]
  \qquad
  \inference{
    \jgmnt[\cEnv;\Theta]{\fForallAtom{\atomv}\formphi}
  }{
    \jgmnt[\cEnv;\Theta]{\formphi \{\atomv \mapsto \atomv'\}}
  }[(\ensuremath{{\fForallAtom{}}^e})]
$$
$$
  \inference{
    \termv \notin \fv[\cEnv;\Theta] &
    \jgmnt[\cEnv;\Theta]{\formphi}
  }{
    \jgmnt[\cEnv;\Theta]{\fForallTerm{\termv}\formphi}
  }[(\ensuremath{{\fForallTerm{}}^i})]
  \qquad
  \inference{
    \jgmnt[\cEnv;\Theta]{\fForallTerm{\termv}\formphi}
  }{
    \jgmnt[\cEnv;\Theta]{\formphi \{\termv \mapsto \termv'\}}
  }[(\ensuremath{{\fForallTerm{}}^e})]
$$
$$
  \inference{
    \jgmnt[\cEnv;\Theta]{\formphi \{\atomv \mapsto \atomv'\}}
    }{
    \jgmnt[\cEnv;\Theta]{\fExistsAtom{\atomv}\formphi}
  }[(\ensuremath{{\fExistsAtom{}}^i})]
  \qquad
  \inference{
    \jgmnt[\cEnv_1;\Theta_1]{\fExistsAtom{\atomv}\formphi} \\
    \jgmnt[\cEnv_2;\Theta_2,\formphi \{\atomv \mapsto \atomv'\}]{\psi} \\
    \atomv' \notin \fv[\cEnv_1 \cup \cEnv_2;\Theta_2 \cup \Theta_2]
    }{
    \jgmnt[\cEnv_1 \cup \cEnv_2;\Theta_2 \cup \Theta_2]{\psi}
  }[(\ensuremath{{\fExistsAtom{}}^e})]
$$
$$
  \inference{
    \jgmnt[\cEnv;\Theta]{\formphi \{\termv \mapsto \termv'\}}
    }{
    \jgmnt[\cEnv;\Theta]{\fExistsTerm{\termv}\formphi}
  }[(\ensuremath{{\fExistsTerm{}}^i})]
  \qquad
  \inference{
    \jgmnt[\cEnv_1;\Theta_1]{\fExistsTerm{\termv}\formphi} \\
    \jgmnt[\cEnv_2;\Theta_2,\formphi \{\termv \mapsto \termv'\}]{\psi} \\
    \termv' \notin \fv[\cEnv_1 \cup \cEnv_2;\Theta_2 \cup \Theta_2]
    }{
    \jgmnt[\cEnv_1 \cup \cEnv_2;\Theta_2 \cup \Theta_2]{\psi}
  }[(\ensuremath{{\fExistsTerm{}}^e})]
$$
To make the framework more flexible we introduce a way for using equivalent formulas:
$$
  \inference{
    \jgmnt[\cEnv;\Theta]{\psi} &
    \jgmnt[\cEnv;\Theta]{\psi \equiv \formphi}
    }{
    \jgmnt[\cEnv;\Theta]{\formphi}
  }[(\ensuremath{Equiv})]
$$
And a way to substitute atoms for atomic expression and variables for terms, if the solver can prove their equality:
$$
  \inference{
    \cjgmnt[\cEnv]{\atomv \ceq \atomexp} &
    \jgmnt[\cEnv;\Theta]{\formphi}
    }{
    \jgmnt[\cEnv\{\atomv \mapsto \atomexp\};\Theta\{\atomv \mapsto \atomexp\}]{\formphi\{\atomv \mapsto \atomexp\}}
  }[(\ensuremath{\mapsto_A})]
\qquad
  \inference{
    \cjgmnt[\cEnv]{\termv \ceq \term} &
    \jgmnt[\cEnv;\Theta]{\formphi}
    }{
    \jgmnt[\cEnv\{\termv \mapsto \term\};\Theta\{\termv \mapsto \term\}]{\formphi\{\termv \mapsto \term\}}
  }[(\ensuremath{\mapsto_T})]
$$
Finall we define induction over term structure,
and thanks to the constraints sublogic we can easily define the notion of
\textit{smaller terms} for inductive hypothesis:
$$
  \inference{
    \jgmnt[\cEnv;\Theta, (\fForallTerm{\termv'} \fCImp{\termv' \cshlt \termv} \formphi(\termv'))]{\formphi(\termv)}
    }{
    \jgmnt[\cEnv;\Theta]{\fForallTerm{\termv} \formphi(\termv)}
  }[(\ensuremath{Induction})]
$$
We also define some axioms about constraint sublogic:
\begin{enumerate}
\item Atoms can be compared in a deterministic fashion,
$$
  \inference{
    }{
    \jgmnt[]{\fForallAtom{\:\atomv,\:\atomv'} (\atomv \ceq \atomv') \vee (\atomv \cneq \atomv')}
  }[(\ensuremath{Axiom_{Compare}})]
$$
\item There are always exists a \textit{fresh} atom,
$$
  \inference{
    }{
    \jgmnt[]{\fForallTerm{\termv} \:\fExistsAtom{\atomv} (\atomv \cfresh \termv)}
  }[(\ensuremath{Axiom_{Fresh}})]
$$
\item We can deduce the structure of a term.
\begin{eqnarray*}
  & \inference{
    }{
    \jgmnt[]{\fForallTerm{\termv} (\fExistsAtom{\atomv}\: \termv = \atomv) \vee (\fExistsAtom{\atomv}\:\fExistsTerm{\termv'}\: \termv = \tbind{\atomv}{\termv'}) }
  }[(\ensuremath{Axiom_{Inversion}})] \\
  & \ensuremath{\vee (\fExistsTerm{\termv_1,\:\termv_2}\: \termv = \tbind{\atomv}{\termv'}) \vee ({symbol}\: \termv) } \\
\end{eqnarray*}
\end{enumerate}

The equivalence relation ($\formphi_1 \equiv \formphi_2$) is a bit complicated
due to the presence of an environment with variable mapping, subkinding, and formulas
with fixpoints, functions, and applications.
Nonetheless, it's simply that - \textit{an equivalence relation} - and it
behaves as expected. We will only highlight the interesting parts.
$$
  \inference{
    \jgmnt[\Gamma; \Sigma]{\formphi_1[\termv_1 \mapsto  \term_1] \equiv \formphi_2[\termv_2 \mapsto \term_2] }
    }{
    \jgmnt[\Gamma; \Sigma]{(\fLamTerm{\termv_1}\formphi_1) \fAppTerm \term_1 \equiv (\fLamTerm{\termv_2}\formphi_2) \fAppTerm \term_2}
  }
$$
Otherwise we compute weak head normal form (up to some \textit{depth}) and recurse on subformulas:
$$
  \inference{
    \jgmnt[\Gamma; \Sigma; S n]{(\fix{P}{\termv}{\kind}\formphi) \fAppTerm \term}
    }{
    \jgmnt[\Gamma; \Sigma, P \mapsto \formphi; n]{\formphi[\termv \mapsto \term]}
  }
$$
Until we reach WHNF computation \textit{depth} or cannot compute the formula further,
we resort to \textit{naive} checking:
$$
  \inference{
    \cjgmnt[\Gamma]{\term_1 \ceq \term_2} &
    \jgmnt[\Gamma; \Sigma]{\formphi_1 \equiv \formphi_2 }
    }{
    \jgmnt[\Gamma; \Sigma]{\formphi_1 \fAppTerm \term_1 \equiv \formphi_2 \fAppTerm \term_2 }
  }
$$
$$
  \inference{
    \termv \notin \fv[\Gamma; \Sigma] \\
    \jgmnt[\Gamma; \Sigma]{\formphi_1[\termv_1 \mapsto \termv] \equiv \formphi_2[\termv_2 \mapsto \termv] }
    }{
    \jgmnt[\Gamma; \Sigma]{\fLamTerm{\termv_1}\formphi_1 \equiv \fLamTerm{\termv_2}\formphi_2 }
  }
$$
$$
  \inference{
    \kind_1 \subkind \kind_2 \\
    \jgmnt[\Gamma; \Sigma]{\formphi_1[P_1 \mapsto P] \equiv \formphi_2[P_2 \mapsto P] }
    }{
    \jgmnt[\Gamma; \Sigma]{\fLamForm{P_1}{\kind_1}{\formphi_1} \equiv \fLamForm{P_2}{\kind_2}{\formphi_2}}
  }
$$
$$
  \inference{
    \kind_1 \subkind \kind_2 &
    P \notin \fv[\Gamma; \Sigma] & \termv \notin \fv[\Gamma; \Sigma] \\
    \jgmnt[\Gamma; \Sigma]{\formphi_1[P_1 \mapsto P, \termv_1 \mapsto \termv] \equiv \formphi_2[P_2 \mapsto P, \termv_2 \mapsto \termv] }
    }{
    \jgmnt[\Gamma; \Sigma]{\fix{P_1}{\termv_1}{\kind_1}\formphi_1 \equiv \fix{P_2}{\termv_2}{\kind_2}\formphi_2}
  }
$$
Note that we allow \textit{different terms} in equivalent formulas as long as
constraints-enviroment $\cEnv$ ensures their equality is provable.
Quantifiers are handled the same way as function above --- as they all a form of bind.
\begin{eqnarray*}
  \inference{
    \jgmnt[\Gamma]{\constr_1 \equiv \constr_2 } &
    \jgmnt[\Gamma; \Sigma]{\formphi_1 \equiv \formphi_2 }
  }{
    \jgmnt[\Gamma; \Sigma]{\fCAnd{\constr_1}\formphi_1 \equiv \fCAnd{\constr_2}\formphi_2 }
  }
\qquad
  \inference{
    \cjgmnt[\Gamma]{\atomv_1 \ceq \atomv_2 } & \cjgmnt[\Gamma]{\term_1 \ceq \term_2 }
  }{
    \jgmnt[\Gamma]{(\atomv_1 \cfresh \term_1) \equiv (\atomv_2 \cfresh \term_2) }
  }
\end{eqnarray*}
To handle formulas with constraints we introduce constraint equivalence relation,
which does nothing more than use solver to check that arguments and constructors
of constraint are equal in the solver sense.
% %%%%%%%%%%%%%%%%%%%%%%%%%%%%%%%%%%%%%%%%%%%%%%%%%%%%%%%%%%%%%%%%%%%%%%%%%%%%%%%%
% \chapter{Model}

% Definition of a model of our logic is bit involved,
% due to presence of subkinding relation.
% We will proceed in two steps.
% First, for each kind $\kind$ we define its \emph{domain} $\kindDom{\kind}$.
% Then we will interpret each kind as a predicate on elements of its domain.
% We fix some Heyting algebra $\PropAlg$
% in which we will interpret propositions.
% Then kind domains are defined in the following way.
% \begin{eqnarray*}
% \kindDom{\kProp}                     & = & \PropAlg \\
% \kindDom{\kind_1 \karrow \kind_2}    & = & \kindDom{\kind_1} \rightarrow \kindDom{\kind_2} \\
% \kindDom{\kForallAtom{\atomv}\kind}  & = & \atomDom          \rightarrow \kindDom{\kind} \\
% \kindDom{\kForallTerm{\termv}\kind}  & = & \termDom          \rightarrow \kindDom{\kind} \\
% \kindDom{\kGuard{\constr}\kind}      & = & \kindDom{\kind}
% \end{eqnarray*}

% %%%%%%%%%%%%%%%%%%%%%%%
% And kind interpretation like this:
% \begin{eqnarray*}
% \termMdl{\kProp}{\tmEnv}                      & = & \{ \bot, \top \} \\
% \termMdl{\kind_1 \karrow \kind_2}{\tmEnv}     & = & \{f \mid \forall P \in \termMdl{\kind_1}{\tmEnv}.\: f(P) \in \termMdl{\kind_2}{\tmEnv} \} \\
% \termMdl{\kForallAtom{\atomv}\kind}{\tmEnv}   & = & \{f \mid \forall A \in \atomDom.\: f(A) \in \termMdl{\kind}{\tmEnv[\atomv \mapsto A]}\} \\
% \termMdl{\kForallTerm{\termv}\kind}{\tmEnv}   & = & \{f \mid \forall T \in \termDom.\: f(T) \in \termMdl{\kind}{\tmEnv[\termv \mapsto T]}\} \\
% \termMdl{\kGuard{\constr}\kind}{\tmEnv}       & = & \{x \mid \tmEnv \vDash \constr \implies x \in \termMdl{\kind}{\tmEnv} \}
% \end{eqnarray*}



% And finally the kind derivation model:
% \begin{eqnarray*}
% \interp{\inferrule{ }{\types{\top}{\kProp}}} & = & \top \\
% \interp{\inference{ }{\types{ \propv}{\Gamma( \propv)}}} & = & \tmEnv(\propv) \\
%     \interp{\inference{}{\types{\constr}{\kProp}}} & = & \texttt{if } \tmEnv \vDash \constr \texttt{ then } \top \texttt{ else } \bot
% \end{eqnarray*}

% \begin{eqnarray*}
%     \interp{
%         \inference{D_1 : \types{\formphi_1 }{\kProp} \\ D_2 : \types{\formphi_2 }{\kProp}
%     }{
%         \types{\formphi_1 \wedge \formphi_2}{\kProp}}
%     } & = & \interp{D_1} \wedge_\PropAlg \interp{D_2} \\
%     \interp{
%         \inference{D_1 : \types{\formphi_1 }{\kProp}  \\ D_2 : \types{\formphi_2 }{\kProp}
%     }{
%         \types{\formphi_1 \vee \formphi_2}{\kProp}}
%     } & = & \interp{D_1} \vee_\PropAlg \interp{D_2} \\
%     \interp{
%         \inference{D_1 : \types{\formphi_1 }{\kProp}  \\ D_2 : \types{\formphi_2 }{\kProp}
%     }{
%         \types{\formphi_1 \Rightarrow \formphi_2}{\kProp}}
%     } & = & \interp{D_1} \Rightarrow_\PropAlg \interp{D_2}
% \end{eqnarray*}

% \begin{eqnarray*}
%     \interp{
%         \inference{D : \types{\formphi}{\kProp}
%     }{
%         \types{\fForallTerm{\termv}\formphi}{\kProp}}
%     } & = & \underset{T \in Term}{\bigwedge} {\interp[{\tmEnv [\termv \mapsto T]}]{D}}  \\
%     \interp{
%         \inference{D : \types{\formphi}{\kProp}
%     }{
%         \types{\fForallAtom{\atomv}\formphi}{\kProp}}
%     } & = & \underset{A \in Atom}{\bigwedge} {\interp[{\tmEnv [\atomv \mapsto A]}]{D}} \\
%     \interp{
%         \inference{D : \types{\formphi}{\kProp}
%     }{
%         \types{\fExistsTerm{\termv} \formphi}{\kProp}}
%     } & = & \underset{T \in Term}{\bigvee}\interp[{\tmEnv [\termv \mapsto T]}]{D} \\
%     \interp{
%         \inference{D : \types{\formphi}{\kProp}
%     }{
%         \types{\fExistsAtom{\atomv} \formphi}{\kProp}}
%     } & = & \underset{A \in Atom}{\bigvee} \interp[{\tmEnv [\atomv \mapsto A]}]{D}
% \end{eqnarray*}

% \begin{eqnarray*}
%    \interp{
%         \inference{D : \types[\Gamma, c]{\formphi}{\kProp}
%     }{
%         \types{[c] \wedge \formphi}{\kProp}}
%     } & = &  \texttt{if } \tmEnv \vDash c \texttt{ then } \interp{D} \texttt{ else } \bot
%     \\
%     \interp{
%         \inference{D : \types[\Gamma, c]{\formphi}{\kProp}
%     }{
%         \types{[c] \Rightarrow \formphi}{\kProp}}
%     } & = &  \texttt{if } \tmEnv \vDash c \texttt{ then } \interp{D} \texttt{ else } \top
% \end{eqnarray*}

% \begin{eqnarray*}
%     \interp{\inference{D : \types{\formphi}{\kind_2}}{\types{\lam[\propv]\formphi}{\kind_1\karr\kind_2}}} & = &
%     \lambda\;(Q : \interp{\kind_1}).\;\interp[{\tmEnv[\propv\mapsto Q]}]{D}
%     \\
%     \interp{\inference{D: \types{\formphi}{\kind}}{\types{\lam[\atomv]\formphi}{\kForallAtom{\atomv}\kind}}} & = &
%     \lambda\;(A : Atom).\;\interp[{\tmEnv[\atomv\mapsto A]}]{D}
%     \\
%     \interp{\inference{D: \types{\formphi}{\kind}}{\types{\lam[\termv]\formphi}{\kForallTerm{\termv}\kind}}} & = &
%     \lambda\;(T : Term).\;\interp[{\tmEnv[\termv\mapsto T]}]{D}
% \end{eqnarray*}

% \begin{eqnarray*}
%     \interp{\inference{D_1: \types{\formphi_1}{\kind'\karr\kind} \\ D_2: \types{\formphi_2}{\kind'}}{\types{\formphi_1\;\formphi_2}{\kind}}} & = &
%     \interp{D_1}\; \interp{D_2}
%     \\
%     \interp{\inference{D: \types{\formphi}{\kForallAtom{\atomv}\kind}}{\types{\formphi(\atomexp)}{\kind\{\atomv \mapsto \atomexp\}}}} & = &
%     \interp{D}\;\interp{\atomexp}
%     \\
%     \interp{\inference{D: \types{\formphi}{\kForallTerm{\termv}\kind}}{\types{\formphi(\term)}{\kind\{\termv \mapsto \term\}}}} & = &
%     \interp{D}\;\interp{\termv}
% \end{eqnarray*}

% %  lim w sensie tw kleenego o fixpoincie
% \begin{eqnarray*}
%     \interp{\inference{
%             D : \types[{\Gamma, X : {\all[z]{[z < \termv']}}\;\kind\subst{z}{\termv'}}]\formphi\kind
%         }{
%             \types{\fix{X}{\termv'}\formphi}{\all[\termv']\kind}
%         }} & = & \lim_{n \rightarrow \infty} f_n
%     \\
%     & & \texttt{ where } f_0(t) = \bot \\
%     & & \texttt{ and } f_{n+1}(t) = \interp[{\tmEnv[X\mapsto f_n, \termv' \mapsto t]}]{D} \\
%     \interp{\inference{
%             D : \types[\Gamma, c]\formphi\kind
%         }{
%             \types{\formphi}{[c]\kind}
%         }} & = & \texttt{if } \tmEnv \vDash c \texttt{ then } \interp{D} \texttt{ else } \text{''}\bot\text{''}
%     \\
%     \interp{\inference{
%             D : \types\formphi\kind \\
%             \rel[\Gamma] \kind \leq \kind'
%         }{
%             \types{\formphi}{\kind'}
%         }} & = &  \interp{D}
% \end{eqnarray*}

% \section{Fundamental Theorem}
% For any formula $\formphi$, any kind $\kind$, and any environment $\Gamma$, for any kind derivation $D : \types\formphi\kind$ under any interpretation $\tmEnv \in \interp[]{\Gamma}$, we have that
% $$
%     \interp{D} \in \interp{\kind}
% $$
% In other words, each kind derivation $D$ has a semantic witness that inhabits the semantic interpretation of $\kind$.
%%%%%%%%%%%%%%%%%%%%%%%%%%%%%%%%%%%%%%%%%%%%%%%%%%%%%%%%%%%%%%%%%%%%%%%%%%%%%%%%
\chapter{Proof assistant}
All the stuff mentioned above has their implementation in OCaml,
in modules \texttt{Solver} and \texttt{SolverEnv},
\texttt{KindChecker} and \texttt{KindCheckerEnv},
\texttt{Proof} and \texttt{ProofEnv}, respectively.

Constraints, kinds, and formulas constructors mirror the grammars we defined
in previous chapters.
Atoms and variables are represented internally by integers (but still are disjoint sets)
--- and their string \textit{names} are kept in the environment and binders
(quantifiers and functions).

Additionally, we provide a \textit{proof assistant} (in module \texttt{Prover}),
that enables the user to conveniently work with \textit{backwards} and incomplete
proof --- inspired by the HOL family of theorem provers.
While simple, it is also powerful and easy to use.

The interface to the Prover provides multiple \textit{tactics}
(functions that manipulate \textit{prover state}) and ways to combine them:
\begin{lstlisting}[language=OCaml]
type goal_env = (string * formula) ProofEnv.env

type goal = goal_env * formula

type prover_state = S_Unfinished of {goal: goal; context: proof_context}
                  | S_Finished of proof

type tactic = prover_state -> prover_state

val proof : goal_env -> formula -> prover_state

val qed : prover_state -> proof

val (|>) : prover_state -> tactic -> prover_state

val (%>) : tactic -> tactic -> tactic

val repeat : tactic -> tactic

val try_tactic : tactic -> tactic
\end{lstlisting}
\newcommand{\hole}{\ensuremath{\bullet}}
We will use $\hole$ to indicate \textit{holes} in the incomplete proofs:
\begin{eqnarray*}
  \text{\lstinline[columns=fixed]{intro}} & & \\
    \jgmnt[\Gamma; \Theta; \Sigma]{\hole\ofkind{\fCImp{\constr}{\formphi}}} & \leadsto & \jgmnt[\Gamma, \constr; \Theta; \Sigma]{\hole\ofkind\formphi} \\
  & & \\
  \text{\lstinline[columns=fixed]{intro' "x"}} & & \\
  \jgmnt[\Gamma; \Theta; \Sigma]{\hole\ofkind{\psi\fImp\formphi}} & \leadsto & \jgmnt[\Gamma; \Theta, \texttt{x} \ofkind\psi ; \Sigma]{\hole\ofkind\formphi} \\
  \jgmnt[\Gamma; \Theta; \Sigma]{\hole\ofkind{\fForallAtom{\atomv}{\formphi}}} & \leadsto & \jgmnt[\Gamma; \Theta; \Sigma, \texttt{x} \ofkind\atomv]{\hole\ofkind\formphi} \\
  \jgmnt[\Gamma; \Theta; \Sigma]{\hole\ofkind{\fForallTerm{\termv}{\formphi}}} & \leadsto & \jgmnt[\Gamma; \Theta; \Sigma, \texttt{x} \ofkind\termv]{\hole\ofkind\formphi} \\
  & & \\
  \text{\lstinline[columns=fixed]{apply_assm "H"}} & & \\
    \jgmnt[\Gamma; \Theta; \Sigma]{\hole\ofkind{{\formphi}}} & \leadsto & \jgmnt[\Gamma; \Theta; \Sigma]{\hole\ofkind\psi} \\
   & \text{when} & (\texttt{H} \ofkind \psi) \in \Theta \\
  & & \\
  \text{\lstinline[columns=fixed]{apply} } (\psi \fImp \formphi)& & \\
  \jgmnt[\Gamma; \Theta; \Sigma]{\hole\ofkind{{\formphi}}} & \leadsto & \jgmnt[\Gamma; \Theta; \Sigma]{\hole\ofkind\psi} \\
  &          & \jgmnt[\Gamma; \Theta; \Sigma]{\hole\ofkind\psi \fImp \formphi} \\
  & & \\
  \text{\lstinline[columns=fixed]{ex_falso} } & & \\
  \jgmnt[\Gamma; \Theta; \Sigma]{\hole\ofkind{{\formphi}}} & \leadsto & \jgmnt[\Gamma; \Theta; \Sigma]{\hole\ofkind\bot} \\
  & & \\
  \text{\lstinline[columns=fixed]{by_solver} }& & \\
  \jgmnt[\Gamma; \Theta; \Sigma]{\hole\ofkind{{\constr}}} & \leadsto & \jgmnt[\Gamma; \Theta; \Sigma]{\constr} \\
   & \text{when} & \cjgmnt[\Gamma]{\constr}\\
  & & \\
  \text{\lstinline[columns=fixed]{compare_atoms "a" "b"} }& & \\
  \jgmnt[\Gamma; \Theta; \Sigma]{\hole\ofkind{{\formphi}}} & \leadsto & \jgmnt[\Gamma; \Theta; \Sigma]{\hole\ofkind{(\atomv \ceq \atomv'\vee\atomv \cneq \atomv')\fImp{\formphi}}} \\
   & \text{when} & (\texttt{a} \ofkind \atomv) \in \Sigma \text{ and } (\texttt{b} \ofkind \atomv') \in \Sigma\\
\end{eqnarray*}
%%%%%%%%%%%%%%%%%%%%%%%%%%%%%%%%%%%%%%%%%%%%%%%%%%%%%%%%%%%%%%%%%%%%%%%%%%%%%%%%
\chapter{Case study: Progress and Preservation of STLC}

The ultimate goal of our work is to create a logic for dealing with variable binding,
and there's no better way to do that than to prove some things about lambda calculus.

We will take a look at simply typed lambda calculus and examine proofs of
its two major properties of \textit{type soundness}: \textit{progress} and \textit{preservation}.
Before we delve into the proofs, let's first establish the needed relations:

\begin{eqnarray*}
\fix{Type}{t}{\kProp} &
  (t \ceq {base}) \\
   \vee & (\fExistsTerm{t_1, t_2}{\fCAnd{\termv \ceq {arrow}\fAppTerm{t_1}\fAppTerm{t_2}}{({Type}\fAppTerm{T_1}) \wedge ({Type}\fAppTerm{t_2}) }}) \\
\fix{InEnv}{env}{\kForallAtom{a}\kForallTerm{t}\kProp}  \fLamAtom{a} \fLamTerm{t} &
  (\fExistsTerm{env'} {env \ceq {cons}\fAppTerm{a}\fAppTerm{t}\fAppTerm{env'}})  \\
  \vee &
  (\fExistsAtom{b}\fExistsTerm{t', env'}\fCAnd{env \ceq {cons}\fAppTerm{b}\fAppTerm{t'}\fAppTerm{env'}}\fCAnd{a \cneq b}({InEnv}\fAppTerm{env'}\fAppTerm{a}\fAppTerm{t}) ) \\
  \cdots & \cdots
\end{eqnarray*}

As one would expect, we will need a lemma about \textit{canonical forms},
which states that all values are of \textit{arrow} type and can be \textit{inversed} into an abstraction term
(since we did not consider any true base types like \texttt{Bool} or \texttt{Int}).
Other lemmas are unimportant boilerplate.

\begin{lstlisting}[language=OCaml]
let empty_contradiction_thm = lambda_thm
    "forall a :atom. forall t :term. (InEnv nil a t) => false"

let typing_terms_thm = lambda_thm
    "forall e env t : term. (Typing e env t) => (Term e)"

let canonical_form_thm = lambda_thm $ concat
      [ "forall v t :term."
      ; " (Value v) =>"
      ; " (Typing v nil t) =>"
      ; " (exists a :atom. exists e :term. [v = lam (a.e)] /\ (Term e))"
      ; " /\ "
      ; " (exists t1 t2 :term. [t = arrow t1 t2])" ]

let subst_exists_thm = lambda_thm $ unwords
      [ "forall a :atom."
      ; "forall v :term. (Value v) =>"
      ; "forall e :term. (Term e) =>"
      ; "exists e' :term. (Sub e a v e')" ]

let progress_thm = lambda_thm
    "forall e t :term. (Typing e nil t) => (Progressive e)"
\end{lstlisting}
Otherwise the proof goes the same way as usual, simple induction over $Typing$.
\begin{lstlisting}[language=OCaml]
let progress =
  proof' progress_thm
  |> by_induction "e0" "IH" %> intro %> destr_intro
  |> intros' ["Ha"; "a"; ""]
    (* e is a var in empty env - contradiction *)
     %> ex_falso
     %> apply_thm_specialized empty_contradiction ["a"; "t"]
     %> assumption
  |> intros' ["Hlam"; "a"; "e_a"; "t1"; "t2"; ""; ""; ""]
    (* e is a lambda - value *)
     %> case "value"
     %> case "lam"
     %> exists' ["a"; "e_a"]
     %> by_solver
     %> apply_thm_specialized typing_terms ["e_a"; "cons a t1 nil"; "t2"]
     %> assumption
  |> intros' ["Happ"; "e1"; "e2"; "t2"; ""; ""]
  (* e is an application - steps *)
  |> add_assumption_parse "He1" "Progressive e1"
  |> add_assumption_parse "He2" "Progressive e2"
  |> destruct_assm "He1"
     %> intros ["Hv1"]
     %> destruct_assm "He2"
     %> intros ["Hv2"] (* Value e1, Value e2 *)
     %> ( add_assumption_thm_specialized "He1lam" canonical_form' ["e1"; "t2"; "t"]
        (* He1lam: [e1 = lam (a.e_a)] /\ (Term e_a) *)
        %> apply_in_assm "He1lam" "Hv1"
        %> apply_in_assm "He1lam" "Happ_1"
        %> destruct_assm' "He1lam" ["a"; "e_a"; ""] )
     %> ( add_assumption_thm_specialized "He_a" subst_exists ["a"; "e2"; "e_a"]
        %> apply_in_assm "He_a" "Hv2"
        %> apply_in_assm "He_a" "He1lam"
        %> destruct_assm' "He_a" ["e_a'"] (* He_a: Sub e_a a e2 e_a' *) )
     %> case "steps"
     %> exists "e_a'"
     %> case "app"
     %> exists' ["a"; "e_a"; "e2"]
     %> by_solver
     %> destruct_goal
     %> apply_assm "Hv2"
     %> apply_assm "He_a"
  |> intros' ["Hs2"; "e2'"] (* Value e1, Steps e2 e2' *)
     %> case "steps"
     %> exists "app e1 e2'"
     %> case "app_r"
     %> exists' ["e1"; "e2"; "e2'"]
     %> by_solver
     %> by_solver
     %> destruct_goal
     %> apply_assm "Hv1"
     %> apply_assm "Hs2"
  |> intros' ["Hs1"; "e1'"] (* Steps e1 *)
     %> case "steps"
     %> exists "app e1' e2"
     %> case "app_l"
     %> exists' ["e1"; "e1'"; "e2"]
     %> by_solver
     %> by_solver
     %> apply_assm "Hs1"
  |> apply_assm_specialized "IH" ["e2"; "t2"] %> by_solver
    %> apply_assm "Happ_2" (* Progressive e2 *)
  |> apply_assm_specialized "IH" ["e1"; "arrow t2 t"] %> by_solver
    %> apply_assm "Happ_1" (* Progressive e1 *)
  |> qed
\end{lstlisting}

\pagebreak

To prove \textit{Preservation}, we will need more lemmas:

1. Substitution lemma:
if term $e$ has a type $t$ in enviroment ${cons}\fAppAtom{a}\fAppTerm{ta}\fAppTerm{env}$,
then we can substitute $a$ for any value $v$ of type $ta$ in $e$ without breaking the typing.
\begin{lstlisting}[language=OCaml]
let sub_lemma_thm = lambda_thm $ concat
      [ "forall e env t :term."
      ; "forall a : atom. forall ta :term."
      ; "forall v e' :term."
      ; " (Typing v env ta) =>"
      ; " (Typing e {cons a ta env} t) =>"
      ; " (Sub e a v e') =>"
      ; " (Typing e' env t)" ]
\end{lstlisting}

2. Swap lemma:
If we have a typing of $e$ in $env$ then we can
swap $a$ with (\textit{fresh enough}) $b$ in both $e$ and $env$ without breaking the typing.
This is particularly useful for mainipulating the abstraction terms ---
we can have any atom we want in the argument position while preserving typing.
\begin{lstlisting}[language=OCaml]
let swap_lambda_typing_lemma_thm = lambda_thm $ unwords
      [ "forall e env t :term. "
      ; "forall a b :atom. forall t' :term. "
      ; " [b # a e] => "
      ; " (Typing {e} {cons a t' env} t) => "
      ; " (Typing {[a b]e} {cons b t' env} t)" ]
\end{lstlisting}

3. Weakening lemma: for any enviroment $env_1$, we can use larger enviroment $env_2$ without breaking the typing.
\begin{lstlisting}[language=OCaml]
let weakening_lemma_thm = lambda_thm $ concat
      [ "forall e env1 t env2 : term."
      ; " (Typing e env1 t) =>"
      ; " (EnvInclusion env1 env2) =>"
      ; " (Typing e env2 t)" ]
\end{lstlisting}

Now to the proof:
\begin{lstlisting}[language=OCaml]
let preservation =
  let contra_var = intros' ["contra"; "_"; ""] %> discriminate in
  let contra_app = intros' ["contra"; "_e1"; "_e2"; "_t2"; ""] %> discriminate in
  let deduce_app_typing =
    destruct_assm "Htyp"
    %> (intros' ["contra"; "_"; ""] %> discriminate)
    %> (intros' ["contra"; "_"; "e_"; "t1"; "t2"; ""] %> discriminate)
    %> intros' ["Happ"; "e_1"; "e_2"; "t2"; ""; ""]
  in
  proof' preservation_thm
  |> by_induction "e0" "IH"
  |> intro %> intro %> intro %> intros ["Htyp"; "Hstep"]
  |> destruct_assm "Hstep"
  |> intros' ["He1"; "e1"; "e1'"; "e2"; ""; ""]
    (* e = app e1 e2, Steps e1 e1' *)
     %> deduce_app_typing
     %> case "app"
     %> exists' ["e1'"; "e2"; "t2"]
     %> by_solver
     %> destruct_goal
     %> apply_assm_specialized "IH" ["e1"; "e1'"; "env"; "arrow t2 t"]
     (* Typing e1 env {arrow t2 t} =>
            Steps e1 e1' => Typing e1' env {arrow t2 t} *)
     %> by_solver
     %> apply_assm "Happ_1"
     %> apply_assm "He1"
     (* Typing e2 env t2 *)
     %> apply_assm "Happ_2"
  |> intros' ["He2"; "v1"; "e2"; "e2'"; ""; ""; ""]
     (* e = app v1 e2, Value e1, Steps e2 e2' *)
     %> deduce_app_typing
     %> case "app"
     %> exists' ["v1"; "e2'"; "t2"]
     %> by_solver
     %> destruct_goal
     (* Typing e1 env {arrow t2 t} *)
     %> apply_assm "Happ_1"
     (* Typing e2 env t2 => Steps e2 e2' => Typing e2' env t2*)
     %> apply_assm_specialized "IH" ["e2"; "e2'"; "env"; "t2"]
     %> by_solver
     %> apply_assm "Happ_2"
     %> apply_assm "He2_2"
  |> intros' ["Hbeta"; "a"; "e_a"; "v"; ""; ""]
     (* e = app (lam (a.e_a)) v, Value v *)
     %> deduce_app_typing
     %> destruct_assm "Happ_1"
     %> contra_var (* e_1 =/= var *)
     %> intros' ["Hlam"; "b"; "e_b"; "t1b"; "t2b"; ""; ""; ""] (* e_1 = b.e_b *)
     %> apply_thm_specialized sub_lemma ["e_a"; "env"; "t"; "a"; "t2"; "v"; "e'"]
     (* Typing v env t2 => Typing e_a {cons a t2 env} t =>
            Sub e_a a v e' => Typing e' env t *)
     %> apply_assm "Happ_2" (* Typing v env t2 *)
     %> compare_atoms "a" "b" (* Typing e_a cons a t2 env t *)
     %> destr_intro
     (* a = b *) %> apply_assm "Hlam_2"
     %> destr_intro (* a =/= b *)
     (* [a # b e_b] => Typing e_b {cons b t2 env} t =>
            Typing {[b a]e_b} {cons a t2 env} t *)
     %> apply_thm_specialized
        swap_lambda_typing ["e_b"; "env"; "t"; "b"; "a"; "t2"]
     %> by_solver
     %> apply_assm "Hlam_2"
     %> apply_assm "Hbeta_2" (* Sub e_a a v e' *)
     %> contra_app (* e_1 =/= app _ _ *)
  |> qed
\end{lstlisting}


%%%%%%%%%%%%%%%%%%%%%%%%%%%%%%%%%%%%%%%%%%%%%%%%%%%%%%%%%%%%%%%%%%%%%%%%%%%%%%%%
\chapter{Conclusion and future work}

\dots

%%%%% BIBLIOGRAFIA

\begin{thebibliography}{1}
\bibitem{example} \ldots
\end{thebibliography}

\end{document}
