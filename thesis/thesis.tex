% Opcje klasy 'iithesis' opisane sa w komentarzach w pliku klasy. Za ich pomoca
% ustawia sie przede wszystkim jezyk oraz rodzaj (lic/inz/mgr) pracy.
\documentclass[english, mgr]{iithesis}

\usepackage[utf8]{inputenc}

%%%%% DANE DO STRONY TYTUŁOWEJ
% Niezaleznie od jezyka pracy wybranego w opcjach klasy, tytul i streszczenie
% pracy nalezy podac zarowno w jezyku polskim, jak i angielskim.
% Pamietaj o madrym (zgodnym z logicznym rozbiorem zdania oraz estetyka) recznym
% zlamaniu wierszy w temacie pracy, zwlaszcza tego w jezyku pracy. Uzyj do tego
% polecenia \fmlinebreak.
\englishtitle   {Nominal logic \fmlinebreak for reasoning about terms \fmlinebreak with variable bindings}
\polishtitle    {Logika dziedzinowa do wnioskowania \fmlinebreak o termach z wiązaniem zmiennych}
\polishabstract {Przedstawiamy logikę dziedzinową do wnioskowania o termach z wiązaniem zmiennych. }
\englishabstract{We describe logic for reasoning about terms with variable bindings.}
\author         {Dominik Gulczyński}
\advisor        {dr Piotr Polesiuk}
\date           {\today}                     % Data zlozenia pracy
% Dane do oswiadczenia o autorskim wykonaniu
\transcriptnum  {299391}                     % Numer indeksu
\advisorgen     {dr. Piotra Polesiuka} % Nazwisko promotora w dopelniaczu
%%%%%

%%%%% WLASNE DODATKOWE PAKIETY
\usepackage{amsmath}
\usepackage{amssymb}
\usepackage{stmaryrd} % For \llbracket
\usepackage{macros}
\usepackage{mathpartir}
\usepackage{listings}

%%%%% WŁASNE DEFINICJE I POLECENIA
% \definecolor{codegreen}{rgb}{0,0.6,0}
% \definecolor{codegray}{rgb}{0.5,0.5,0.5}
% \definecolor{codepurple}{rgb}{0.58,0,0.82}
% \definecolor{backcolour}{rgb}{0.95,0.95,0.92}
% \lstdefinestyle{ocamlstyle}{
%     commentstyle=\color{codegreen},
%     keywordstyle=\color{magenta},
%     numberstyle=\tiny\color{codegray},
%     stringstyle=\color{codepurple},
%     basicstyle=\ttfamily\footnotesize,
%     breakatwhitespace=false,
%     breaklines=true,
%     captionpos=b,
%     keepspaces=true,
%     numbers=left,
%     numbersep=5pt,
%     showspaces=false,
%     showstringspaces=false,
%     showtabs=false,
%     tabsize=2
% }
% \lstset{style=ocamlstyle}
%%%%%

\begin{document}

%%%%%%%%%%%%%%%%%%%%%%%%%%%%%%%%%%%%%%%%%%%%%%%%%%%%%%%%%%%%%%%%%%%%%%%%%%%%%%%%
\chapter{Introduction}

Problem statement

Motivation

Related work (logiki nominalne i permutacje)


Contributions

% Przykłady machania rękoma nad wiązaniem zmiennych w innych tekstach jako motywacja do tej pracy

%%%%%%%%%%%%%%%%%%%%%%%%%%%%%%%%%%%%%%%%%%%%%%%%%%%%%%%%%%%%%%%%%%%%%%%%%%%%%%%%
\chapter{Terms and constraints}
% Okreli odbiorce i do niego pisa

In classical first-order logic, terms are built from variables and
applications of functional symbols to other terms.
In this work we expand terms with expressions closely resembling the syntax
of lambda calculus, aiming to provide a flexible framework for reasoning about
the lambda calculus and its derivations.

To this end we introduce an infinite set of \textit{atoms}
(denoted by lower-case letters), representing the bound variables in terms
--- i.e. the variables in the sense of lambda calculus.
That set is disjount with the set of variables commonly found in first-order logic,
which from now on we will call \textit{variables} (and denote by uppper-case letters)
as apposed to \textit{atoms}.

Terms are given by the following grammar: \\
\begin{tabular}{rclr}
    $\perm$    & $::=$ & $\permid
                   \bnfor \permswp{\atomexp}{\atomexp}{\perm}$
        & (permutations) \\
    $\atomexp$ & $::=$ & $\perm \apperm \atomv$
        & (atom expressions) \\
    $\term$    & $::=$ & $\atomexp
                   \bnfor \perm \apperm \termv
                   \bnfor \tbind{\atomexp} \term
                   \bnfor \term \tapp \term
                   \bnfor \symb$
        & (terms)
\end{tabular}
% Termy są czysto sytnatkyczne, opisać ich znaczenie, a model w innym rozdziale

It is important to note that terms do not incorporate any inherent notions of
computation, reduction, or binding.
These expressions simply \textit{look} like the lambda calculus but they lack
the operational semantics of it.
However, the intuitions associated with such expressions are not unfounded.
We will observe their practical application in the sublogic of constraints
that we define on top of terms to reason about
notions of \textit{freshness}, \textit{variable binding} and \textit{structural}
order and its logical model.

Constraints are given by the following grammar:
\\
\begin{tabular}{rclr}
    $\constr$  & $::=$ & $\atomexp \cfresh \term
                   \bnfor \term \ceq \term
                   \bnfor \term \csheq \term
                   \bnfor \term \cshlt \term$
        & (constraints)
\end{tabular}
\\
with following semantics:
\begin{eqnarray*}
  \atomexp \cfresh \term & \text{---} & \text{
    atom $\atomexp$ is fresh in term $\term$, i.e. does not occur in $\term$
    as a free variable
  } \\
  \term_1 \ceq \term_2 & \text{---} &\text{
    terms $\term_1$ and $\term_2$ are alpha-equivalent
  } \\
  \term_1 \csheq \term_2 & \text{---} &\text{
    terms $\term_1$ and $\term_2$ possess an identical shape,
  } \\
  & & \text{
    i.e. after erasing all atoms, terms $\term_1$ and $\term_2$ would be equal
  } \\
  \term_1 \cshlt \term_2 & \text{---} &\text{
    shape of term $\term_1$ is structurally smaller than the shape of term
    $\term_2$,
  } \\
  & & \text{
    i.e. after erasing all atoms $\term_1$ would be equal to some
    subterm of $\term_2$
  } \\
\end{eqnarray*}

We use metavariable $\cEnv$ for finite sets of constraints.


TODO: Explain Model

\begin{tabular}{rclr}
$\semterm$ & $::=$ & $\sematom
               \bnfor n
               \bnfor \stbind \semterm
               \bnfor \semterm \stapp \semterm
               \bnfor \symb$
    & (semantic terms) \\
$\shape$   & $::=$ & $\shatom
               \bnfor \shbind \shape
               \bnfor \shape \shapp \shape
               \bnfor \symb$
    & (semantic shapes)
\end{tabular}

\begin{eqnarray*}
  \termMdl{\perm \apperm \atomv}{\tmEnv} & = &
    \permMdl{\perm}{\tmEnv}(\tmEnv(\atomv)) \\
  \termMdl{\perm \apperm \termv}{\tmEnv} & = &
    \permMdl{\perm}{\tmEnv}(\tmEnv(\termv)) \\
  \termMdl{\tbind{\atomexp} \term}{\tmEnv} & = &
    \stbind (\termMdl{\term}{\tmEnv} \shiftIdx)
      \subst{\termMdl{\atomexp}{\tmEnv}}{0} \\
  \termMdl{\term_1 \tapp \term_2}{\tmEnv} & = &
    \termMdl{\term_1}{\tmEnv} \stapp \termMdl{\term_2}{\tmEnv} \\
  \termMdl{\symb}{\tmEnv} & = & \symb
\end{eqnarray*}

\begin{eqnarray*}
  \shapeof{\sematom}                     & = & \shatom \\
  \shapeof{n}                            & = & \shatom \\
  \shapeof{\stbind \semterm}             & = & \shbind \shapeof{\semterm} \\1
  \shapeof{\semterm_1 \stapp \semterm_2} & = &
    \shapeof{\semterm_1} \shapp \shapeof{\semterm_2}
\end{eqnarray*}

\begin{eqnarray*}
  \tmEnv \vDash \term_1 \ceq \term_2 & \textrm{iff} &
    \termMdl{\term_1}{\tmEnv} = \termMdl{\term_2}{\tmEnv} \\
  \tmEnv \vDash \atomexp \cfresh \term & \textrm{iff} &
    \termMdl{\atomexp}{\tmEnv} \notin
      \mathsf{FreeAtoms}(\termMdl{\term}{\tmEnv}) \\
  \tmEnv \vDash \term_1 \csheq \term_2 & \textrm{iff} &
    \shapeof{\termMdl{\term_1}{\tmEnv}} = \shapeof{\termMdl{\term_2}{\tmEnv}} \\
  \tmEnv \vDash \term_1 \cshlt \term_2 & \textrm{iff} &
    \shapeof{\termMdl{\term_1}{\tmEnv}} \textrm{ is a strict subshape of }
      \shapeof{\termMdl{\term_2}{\tmEnv}}
\end{eqnarray*}

We write $\tmEnv \vDash \cEnv$ iff for all $\constr \in \cEnv$
we have $\tmEnv \vDash \constr$.
We write $\cEnv \vDash \constr$ iff
for every $\tmEnv$ such that $\tmEnv \vDash \cEnv$
we have $\tmEnv \vDash \constr$.


With this model in mind we will se that there exists a decidibile algorithm
for determining whether C1,...,Cn |= C0,
i.e. a deterministic way of checking if constraints c1, ..., cn imply c0.
We present such algorithm in the next chapter.


%%%%%%%%%%%%%%%%%%%%%%%%%%%%%%%%%%%%%%%%%%%%%%%%%%%%%%%%%%%%%%%%%%%%%%%%%%%%%%%%
\chapter{Constraint solver}
Bird's eye view: Solver breaks down constraints (on both sides of the turnstile)
to irreducible components that are solved easily.


At the core of our work lies the Solver
--- the algorithm of resolving the constraints.
Given a list of assumptions $\constr_1, \dots, \constr_n$ it checks whether
given goal $\constr_0$ holds.
In other words it is an algorithm that verifies whether,
for every possible substitution of closed terms (in terms of variables, not atoms)
for variables in $\constr_0, \constr_1, \dots, \constr_n$ such that the constraints
$\constr_1, \dots, \constr_n$ are satisfied, $\constr_0$ is also satisfied.

For convenience and effectiveness of implementation, the Solver works with
constraints a little different constraints (although not more expressive)
than those occuring in formulas and kinds,
main difference being use of \textit{shapes} instead of terms for shape constraints.
Solver contraints and shapes are given by the following grammar: \\
\begin{tabular}{rclr}
  $\sconstr$ & $::=$ & $\atomexp \cfresh \term
  \bnfor \term \ceq \term
  \bnfor \shape \csheq \shape
  \bnfor \shape \cshlt \shape$
      & (solver constraints) \\
  $\shape$      & $::=$ & $\shatom
                 \bnfor  \termv
                 \bnfor \shbind \shape
                 \bnfor \shape \tapp \shape
                 \bnfor \symb$
      & (shapes)
\end{tabular}\\
Solver erases atoms from terms in shape contstraints, effectively transforming
them from \textit{constraints} to \textit{solver constraints}.

We add another environment $\icEnv$ to distinguish between the
potentially-reducible assumptions in $\cEnv$.
For convenience we will write $\atomv \cneq \atomexp$ instead of $\atomv \cfresh \atomexp$ as it gives
good intuition of atom freshness implying inequality
and for $\atomexp = \pi \atomv$ we will write $\atomexp \cfresh \term$ meaning
$\atomv \cfresh \pi^{-1} \term$.
Irreducible constraints are:
\begin{eqnarray*}
  \atomv_1 \cneq \atomv_2 & \text{---} & \text{
    atoms $\atomv_1$ and $\atomv_2$ are different
  } \\
  \atomv   \cfresh \termv   & \text{---} & \text{
    atom $\atomv$ is fresh in variable $\termv$
  } \\
  \termv_1 \csheq  \termv_2 & \text{---} & \text{
    variables $\termv_1$ and $\termv_2$ posses the same shape
  } \\
  \termv   \csheq  \term    & \text{---} & \text{
    variable $\termv$ has a shape of term $\term$
  } \\
  \term    \cshlt  \termv   & \text{---} & \text{
    term $\term$ strictly subshapes variable $\termv$
  } \\
\end{eqnarray*}

After all the constraints are reduced to such simple constraints
we reduce the goal-constraint and repeat the reduction procedure on
new assumptions and goal. We either arrive on a contradictory environment or
all the assumptions and goal itself are reduced to irreducible constraints which
is as simple as checking if the goal occurs on the left side of the turnstile.

$$
\inferrule{
  \constr \in \icEnv
}{
  \cEnv ; \icEnv \vDash \constr
}
$$
Decidability of atom equality plays an important role in the reduce
procedure:
%%%%%%%%%%%%%%%%%%%%%%%%%%%%%%%%%%%%%%%%%%%%%%%%%%%%%%%%%%%%%%%%%%%%%%%%%%%%%%%%
% solver goal eq
%%%%%%%%%%%%%%%%%%%%%%%%%%%%%%%%%%%%%%%%%%%%%%%%%%%%%%%%%%%%%%%%%%%%%%%%%%%%%%%%
$$
\inferrule{
  \cEnv ; \icEnv \vDash \atomv = \perm^{-1} \atomexp
}{
  \cEnv ; \icEnv \vDash \perm \atomv = \atomexp
}
\qquad
\inferrule{
  \atomv \cneq \atomexp_1, \atomv \cneq \atomexp_2, \cEnv ; \icEnv \vDash \atomv     = \atomexp \\
  \atomv \ceq  \atomexp_1, \atomv \cneq \atomexp_2, \cEnv ; \icEnv \vDash \atomexp_2 = \atomexp \\
  \atomv \ceq  \atomexp_2, \cEnv ; \icEnv \vDash \atomexp_1 = \atomexp
}{
  \cEnv ; \icEnv \vDash \atomv = \permswp{\atomexp_1}{\atomexp_2}{}\atomexp
}
$$
$$
\inferrule{
  \cEnv ; \icEnv \vDash \pi \text{ idempotent on } \termv
}{
  \cEnv ; \icEnv \vDash \termv = \pi \termv
}
\qquad
\inferrule{
  \cEnv ; \icEnv \vDash \termv_1 = \pi_1^{-1} \pi_2 \termv_2
}{
  \cEnv ; \icEnv \vDash \pi_1 \termv_1 = \pi_2 \termv_2
}
$$
$$
\inferrule{
  \cEnv ; \icEnv \vDash \atomexp_1 \cfresh \term_2
  \\
  \cEnv ; \icEnv \vDash \term_1 = \permswp{\atomexp_1}{\atomexp_2} \term_2
}{
  \cEnv ; \icEnv \vDash \tbind{\atomexp_1} \term_1 = \tbind{\atomexp_2} \term_2
}
\qquad
\inferrule{
  \cEnv ; \icEnv \vDash \term_1 = \term_2
  \\
  \cEnv ; \icEnv \vDash \term_1' = \term_2'
}{
  \cEnv ; \icEnv \vDash \term_1 \term_1' = \term_2 \term_2'
}
$$
$$\inferrule{
}{
  \cEnv ; \icEnv \vDash \atomv = \atomv
}
\qquad
\inferrule{
}{
  \cEnv ; \icEnv \vDash \termv = \termv
}
\qquad
\inferrule{
}{
  \cEnv ; \icEnv \vDash \symb = \symb
}
$$
%%%%%%%%%%%%%%%%%%%%%%%%%%%%%%%%%%%%%%%%%%%%%%%%%%%%%%%%%%%%%%%%%%%%%%%%%%%%%%%%
% solver permutation idempotent
%%%%%%%%%%%%%%%%%%%%%%%%%%%%%%%%%%%%%%%%%%%%%%%%%%%%%%%%%%%%%%%%%%%%%%%%%%%%%%%%
$$
\inferrule{
  \forall \atomv \in \pi.\;
    \cEnv ; \icEnv \vDash \atomv = \pi \atomv \;\vee\;
    \cEnv ; \icEnv \vDash \atomv \cfresh \termv
  }{
  \cEnv ; \icEnv \vDash \pi \text{ idempotent on } \termv
}
$$
%%%%%%%%%%%%%%%%%%%%%%%%%%%%%%%%%%%%%%%%%%%%%%%%%%%%%%%%%%%%%%%%%%%%%%%%%%%%%%%%
% solver goal fresh
%%%%%%%%%%%%%%%%%%%%%%%%%%%%%%%%%%%%%%%%%%%%%%%%%%%%%%%%%%%%%%%%%%%%%%%%%%%%%%%%
$$\inferrule{
  \atomv_1 \cneq \atomv_2 \in \icEnv
}{
  \cEnv ; \icEnv \vDash \atomv_1 \cfresh \atomv_2
}
\qquad
\inferrule{
  \atomv \cneq \atomexp_1, \atomv \cneq \atomexp_2, \cEnv ; \icEnv \vDash \atomv     \cfresh \atomexp \\
  \atomv \ceq  \atomexp_1, \atomv \cneq \atomexp_2, \cEnv ; \icEnv \vDash \atomexp_1 \cfresh \atomexp \\
                          \atomv \ceq  \atomexp_2 , \cEnv ; \icEnv \vDash \atomexp_2 \cfresh \atomexp
}{
  \cEnv ; \icEnv \vDash \atomv \cfresh \permswp{\atomexp_1}{\atomexp_2}{} \atomexp
}
$$
$$\inferrule{
  \atomv \cfresh \termv \in \icEnv
}{
  \cEnv ; \icEnv \vDash \atomv \cfresh \termv
}
\qquad
\inferrule{
  \atomv \cneq \atomexp_1, \atomv \cneq \atomexp_2, \cEnv ; \icEnv \vDash \atomv     \cfresh \pi \termv \\
  \atomv \ceq  \atomexp_1, \atomv \cneq \atomexp_2, \cEnv ; \icEnv \vDash \atomexp_1 \cfresh \pi \termv \\
                          \atomv \ceq  \atomexp_2 , \cEnv ; \icEnv \vDash \atomexp_2 \cfresh \pi \termv
}{
  \cEnv ; \icEnv \vDash \atomv \cfresh \permswp{\atomexp_1}{\atomexp_2}{\pi} \termv
}
$$
$$\inferrule{
  \atomv \cfresh \atomexp, \cEnv ; \icEnv \vDash \atomv \cfresh \term
}{
  \cEnv ; \icEnv \vDash \atomv \cfresh \tbind{\atomexp}{\term}
}
\qquad
\inferrule{
  \cEnv ; \icEnv \vDash \atomv \cfresh \term_1 \\
  \cEnv ; \icEnv \vDash \atomv \cfresh \term_2
}{
  \cEnv ; \icEnv \vDash \atomv \cfresh \term_1 \term_2
}
\qquad
\inferrule{
}{
  \cEnv ; \icEnv \vDash \atomv \cfresh \symb
}
$$
%%%%%%%%%%%%%%%%%%%%%%%%%%%%%%%%%%%%%%%%%%%%%%%%%%%%%%%%%%%%%%%%%%%%%%%%%%%%%%%%
% solver goal shape
%%%%%%%%%%%%%%%%%%%%%%%%%%%%%%%%%%%%%%%%%%%%%%%%%%%%%%%%%%%%%%%%%%%%%%%%%%%%%%%%
$$\inferrule{
  \termv_1 \csheq \termv_2 \in \icEnv
}{
  \cEnv ; \icEnv \vDash \termv_1 \csheq \termv_2
}
\qquad
\inferrule{
  \termv  \csheq \shape' \in \icEnv \\
  \cEnv ; \icEnv \vDash \shape'  \csheq \shape
}{
  \cEnv ; \icEnv \vDash \termv  \csheq \shape
}
$$
$$\inferrule{
  \cEnv ; \icEnv \vDash \shape_1 \csheq \shape_2
}{
  \cEnv ; \icEnv \vDash \shbind \shape_1 \csheq \shbind \shape_2
}
\qquad
\inferrule{
  \cEnv ; \icEnv \vDash \shape_1 \csheq \shape_2 \\
  \cEnv ; \icEnv \vDash \shape_1' \csheq \shape_2'
}{
  \cEnv ; \icEnv \vDash \shape_1 \shape_1' \csheq \shape_2 \shape_2'
}
\qquad
\inferrule{
}{
  \cEnv ; \icEnv \vDash \symb \csheq \symb
}
$$
%%%%%%%%%%%%%%%%%%%%%%%%%%%%%%%%%%%%%%%%%%%%%%%%%%%%%%%%%%%%%%%%%%%%%%%%%%%%%%%%
% solver goal subshape
%%%%%%%%%%%%%%%%%%%%%%%%%%%%%%%%%%%%%%%%%%%%%%%%%%%%%%%%%%%%%%%%%%%%%%%%%%%%%%%%
$$\inferrule{
  \shape_2 \cshlt \termv \in \icEnv \\
  \cEnv ; \icEnv \vDash \shape_2 \csheq \termv
}{
  \cEnv ; \icEnv \vDash \shape_1 \cshlt \termv
}
\qquad
\inferrule{
  \shape_2 \cshlt \termv \in \icEnv \\
  \cEnv ; \icEnv \vDash \shape_2 \cshlt \termv
}{
  \cEnv ; \icEnv \vDash \shape_1 \cshlt \termv
}
$$
$$\inferrule{
  \cEnv ; \icEnv \vDash \shape_1 \csheq \shape_2
}{
  \cEnv ; \icEnv \vDash \shape_1 \cshlt \shbind \shape_2
}
\qquad
\inferrule{
  \cEnv ; \icEnv \vDash \shape_1 \cshlt \shape_2
}{
  \cEnv ; \icEnv \vDash \shape_1 \cshlt \shbind \shape_2
}
$$
$$\inferrule{
  \cEnv ; \icEnv \vDash \shape_1 \csheq \shape_2
}{
  \cEnv ; \icEnv \vDash \shape_1 \cshlt \shape_2 \shape_2'
}
\qquad
\inferrule{
  \cEnv ; \icEnv \vDash \shape_1 \csheq \shape_2'
}{
  \cEnv ; \icEnv \vDash \shape_1 \cshlt \shape_2 \shape_2'
}
\qquad
\inferrule{
  \cEnv ; \icEnv \vDash \shape_1 \cshlt \shape_2
}{
  \cEnv ; \icEnv \vDash \shape_1 \cshlt \shape_2 \shape_2'
}
\qquad
\inferrule{
  \cEnv ; \icEnv \vDash \shape_1 \cshlt \shape_2'
}{
  \cEnv ; \icEnv \vDash \shape_1 \cshlt \shape_2 \shape_2'
}
$$

Define state of the solver by triple $(\cEnv, \icEnv, \constr_0)$ and such
ordering of the states:
\begin{enumerate}
  \item Number of distinct variables in $\cEnv$, $\icEnv$,$ \constr_0$.
  \item Depth of $\constr_0$.
  \item Number of assumptions of given depth in $\cEnv$ and $\icEnv$.
  \item Number of assumptions of given depth in $\cEnv$.
\end{enumerate}

Then by analysing each rule we can see the reductions always arrive in a smaller
state.

\section{Implementation}

TODO: Description of the special environment  $\icEnv$ and occurs check

%%%%%%%%%%%%%%%%%%%%%%%%%%%%%%%%%%%%%%%%%%%%%%%%%%%%%%%%%%%%%%%%%%%%%%%%%%%%%%%%
\chapter{Higher Order Logic}

\section{Kinds}
\section{Subkinding}

\section{Formulas}

\section{Fixpoint}


\dots

%%%%%%%%%%%%%%%%%%%%%%%%%%%%%%%%%%%%%%%%%%%%%%%%%%%%%%%%%%%%%%%%%%%%%%%%%%%%%%%%
\chapter{Proof theory}

\dots

%%%%%%%%%%%%%%%%%%%%%%%%%%%%%%%%%%%%%%%%%%%%%%%%%%%%%%%%%%%%%%%%%%%%%%%%%%%%%%%%
\chapter{Proof assistant}

\dots

%%%%%%%%%%%%%%%%%%%%%%%%%%%%%%%%%%%%%%%%%%%%%%%%%%%%%%%%%%%%%%%%%%%%%%%%%%%%%%%%
\chapter{Example: Progress and Preservation of STLC}

\dots

%%%%%%%%%%%%%%%%%%%%%%%%%%%%%%%%%%%%%%%%%%%%%%%%%%%%%%%%%%%%%%%%%%%%%%%%%%%%%%%%
\chapter{Conclusion and future work}

\dots

%%%%% BIBLIOGRAFIA

\begin{thebibliography}{1}
\bibitem{example} \ldots
\end{thebibliography}

\end{document}
